% This file was created automatically from unaryop.msk.
% DO NOT EDIT!
These are operations that can be performed over graphs.

\>LineGraph( <G> ) O

Returns the line graph of graph <G>. The line graph is the
intersection graph of the edges of <G>, <i.e.> the vertices of
$L(G)$ are the edges of <G> two of them being adjacent iff they
are incident. 

\beginexample 
gap> G:=Tetrahedron;
Graph( Category := SimpleGraphs, Order := 4, Size := 6, Adjacencies :=
[ [ 2, 3, 4 ], [ 1, 3, 4 ], [ 1, 2, 4 ], [ 1, 2, 3 ] ] )
gap> LineGraph(G);
Graph( Category := SimpleGraphs, Order := 6, Size := 12, Adjacencies :=
[ [ 2, 3, 4, 5 ], [ 1, 3, 4, 6 ], [ 1, 2, 5, 6 ], [ 1, 2, 5, 6 ],
  [ 1, 3, 4, 6 ], [ 2, 3, 4, 5 ] ] )
\endexample


LineGraph(\xygraph{ %/r10mm/:
 \bullet (-[u(.5)]\bullet,-[dr(.5)]\bullet,-[dl(.5)]\bullet),
 [u(.5)] ([d(.5)][dr(.5)]-?, [d(.5)][dl(.5)]-?),
 [dr(.5)]-[l]
}) = 
\xygraph{[u(.5)]
 \bullet-[r]\bullet-[dl(.5)]\bullet-[dl(.5)]\bullet
   -[ul(.5)]\bullet-[ul(.5)]\bullet-[r]-[dl(.5)]-[r]-[ul(.5)],
 [r(.5)][ur(.5)] ([ll]-@(ur,ul)?, [dl]-@(r,d)?),
 [l(.5)][ul(.5)] ([dr]-@(l,d)?)
}

\>ComplementGraph( <G> ) O

Computes the complement of graph <G>. The complement of a graph is
created as follows:
Create a graph <G'> with same vertices of <G>. For each <x>, <y>
$\in$ <G> if <x> $\nsim$ <y> in <G> then <x> $\sim$ <y> in <G'>

\beginexample 
gap> G:=ClawGraph;
Graph( Category := SimpleGraphs, Order := 4, Size := 3, Adjacencies :=
[ [ 2, 3, 4 ], [ 1 ], [ 1 ], [ 1 ] ] )
gap> ComplementGraph(G);
Graph( Category := SimpleGraphs, Order := 4, Size := 3, Adjacencies :=
[ [  ], [ 3, 4 ], [ 2, 4 ], [ 2, 3 ] ] )
\endexample


ComplementGraph(\xymatrix{
   {\bullet} \ar@{-}[r] \ar@{-}[d] & {\bullet} \ar@{-}[d]\\
   {\bullet} \ar@{-}[r] & {\bullet} 
}  
) = 
\xymatrix{
   {\bullet} \ar@{-}[rd] & {\bullet} \ar@{-}[dl]\\
   {\bullet} & {\bullet} 
}  

\>QuotientGraph( <G>, <P> ) O

Returns the quotient graph of graph <G> given a partition of edges
<P>. The quotient graph is the intersection graph of the subsets
of vertives given by partition <P>, <i.e.> the vertices of the quotient
graph are sets of vertices given by partition <P> two of them
being adjacent iff any two of the vertices in the sets are
adjacent in <G>. 

\beginexample 
gap> G:=HouseGraph;
Graph( Category := SimpleGraphs, Order := 5, Size := 6, Adjacencies :=
[ [ 2, 3 ], [ 1, 3, 4 ], [ 1, 2, 5 ], [ 2, 5 ], [ 3, 4 ] ] )
gap> QuotientGraph(G,[[1,2],[4,5]]);
Graph( Category := SimpleGraphs, Order := 3, Size := 3, Adjacencies :=
[ [ 2, 3 ], [ 1, 3 ], [ 1, 2 ] ] )
\endexample


\objectmargin={1pt}
QuotientGraph(\xygraph{
 \bullet,-[dl(.5)]\bullet-[d(.5)]\bullet-[r]\bullet-[u(.5)]\bullet-[l]\bullet,
  -[dr(.5)]\bullet
}, [2,3],[4,5]) =
\xygraph{[u(.2)]
 \bullet-[dr(.5)]\bullet-[l]\bullet-[ur(.5)]
}

