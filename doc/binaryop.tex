% This file was created automatically from binaryop.msk.
% DO NOT EDIT!
These are binary operations that can be performed over graphs.

\>BoxProduct( <G>, <H> ) O

Returns the box product of two graphs <G> and <H> (also called
cartesian product), <G> $\square$ <H>.

The box product is calculated as follows:

For each pair of vertices $g \in G, h \in H$ we create a vertex
$(g,h).$ Given two such vertices $(g,h)$ and $(g',h')$ they are
adjacent <iff> $g = g'$ and $h \sim h'$ or $g \sim g'$ and $h = h'.$

\beginexample
gap> G:=AdjMatrixGraph([[false,true],[true,false]]);
Graph( Category := SimpleGraphs, Order := 2, Size := 1, Adjacencies :=
[ [ 2 ], [ 1 ] ] )
gap> BoxProduct(G,G);
Graph( Category := SimpleGraphs, Order := 4, Size := 4, Adjacencies :=
[ [ 2, 3 ], [ 1, 4 ], [ 1, 4 ], [ 2, 3 ] ] )
\endexample



$$
\xymatrix{
   {\bullet} \\
   {\bullet} \ar@{-}[u]
} 
\square
\xymatrix{
   {\bullet} \\
   {\bullet} \ar@{-}[u]
}
=
\xymatrix{
   {\bullet} \ar@{-}[r] \ar@{-}[d] & {\bullet} \ar@{-}[d]\\
   {\bullet} \ar@{-}[r] & {\bullet} 
}  
$$

\>TimesProduct( <G>, <H> ) O

Returns the times product of two graphs <G> and <H>, <G> $\times$ <H>.

The times product is computed as follows:

For each pair of vertices $g \in G, h \in H$ we create a vertex
$(g,h).$ Given two such vertices $(g,h)$ and $(g',h')$ they are
adjacent <iff> $g \sim g'$ and $h \sim h'.$

\beginexample
gap> G:=AdjMatrixGraph([[false,true],[true,false]]);
Graph( Category := SimpleGraphs, Order := 2, Size := 1, Adjacencies :=
[ [ 2 ], [ 1 ] ] )
gap> TimesProduct(G,G);
Graph( Category := SimpleGraphs, Order := 4, Size := 2, Adjacencies :=
[ [ 4 ], [ 3 ], [ 2 ], [ 1 ] ] )
\endexample



$$
\xymatrix{
   {\bullet} \\
   {\bullet} \ar@{-}[u]
} 
\times
\xymatrix{
   {\bullet} \\
   {\bullet} \ar@{-}[u]
}
=
\xymatrix{
   {\bullet} \ar@{-}[rd] & {\bullet} \ar@{-}[dl]\\
   {\bullet} & {\bullet} 
}  
$$

\>BoxTimesProduct( <G>, <H> ) O

Returns the box times product of two graphs <G> and <H>, <G>
boxtimes <H>. 

The box times product is calculated as follows:

For each pair of vertices $g \in G, h \in H$ we create a vertex
$(g,h).$ Given two such vertices $(g,h)$ and $(g',h')$ such that
$(g,h) \neq (g',h')$ they are adjacent <iff> $g \simeq g'$ and $h
\simeq h'.$ 

\beginexample
gap> G:=AdjMatrixGraph([[false,true],[true,false]]);
Graph( Category := SimpleGraphs, Order := 2, Size := 1, Adjacencies :=
[ [ 2 ], [ 1 ] ] )
gap> G:=BoxTimesProduct(G,G);
Graph( Category := SimpleGraphs, Order := 4, Size := 6, Adjacencies :=
[ [ 2, 3, 4 ], [ 1, 3, 4 ], [ 1, 2, 4 ], [ 1, 2, 3 ] ] )
\endexample



$$
\xymatrix{
   {\bullet} \\
   {\bullet} \ar@{-}[u]
} 
\boxtimes
\xymatrix{
   {\bullet} \\
   {\bullet} \ar@{-}[u]
}
=
\xymatrix{
   {\bullet} \ar@{-}[r] \ar@{-}[d] \ar@{-}[rd] & {\bullet}
   \ar@{-}[d] \ar@{-}[dl] \\ 
   {\bullet} \ar@{-}[r] & {\bullet} 
}  
$$

In the previous examples $k^2$ (<i.e.> the complete graph or order
two) was chosen because it better pictures how the operators work.

\>DisjointUnion( <G>, <H> ) O

Returns the DisjointUnion of two graphs <G> and <H>, <G> $\dot{\cup}$ <H>.

A disjoint union of graphs is obtained by combining both graphs in a
disjoint graph.

\beginexample
gap> G:=RandomGraph(4);
Graph( Category := Graphs, Order := 4, Size := 8, Adjacencies :=
[ [ 3, 4 ], [ 4 ], [ 1, 2, 3, 4 ], [ 2 ] ] )
gap> H:=RandomGraph(4);
Graph( Category := Graphs, Order := 4, Size := 8, Adjacencies :=
[ [ 3, 4 ], [ 4 ], [ 1, 2, 3, 4 ], [ 2 ] ] )
gap> DisjointUnion(G,H);
Graph( Category := Graphs, Order := 4, Size := 8, Adjacencies :=
[ [ 3, 4 ], [ 4 ], [ 1, 2, 3, 4 ], [ 2 ] ] )
\endexample



$$
\xymatrix{
   {\bullet} \ar@{-}[d] & {\bullet} \ar@{-}[l] \\
   {\bullet} \ar@{-}[ur]
} 
\dot{\cup}
\xymatrix{
   {\bullet} \\
   {\bullet} \ar@{-}[u]
}
=
\xymatrix{
   {\bullet} \ar@{-}[d] & {\bullet} \ar@{-}[l] \\
   {\bullet} \ar@{-}[ur]
} 
\xymatrix{
   {\bullet} \\
   {\bullet} \ar@{-}[u]
}
$$

\>Join( <G>, <H> ) O

Returns the result of joining graph <G> and <H>, <G> + <H>.

Joining graphs is computed as follows:

First, we obtain the disjoint union of graphs <G> and <H>. Second,
for each vertex $g \in G$ we add an edge to each vertex $h \in H.$
Finally, for each vertex $g \in G$ we add an edge to each vertex $h
\in H.$ 

\beginexample
gap> G:=RandomGraph(4);
Graph( Category := Graphs, Order := 4, Size := 8, Adjacencies :=
[ [ 3, 4 ], [ 4 ], [ 1, 2, 3, 4 ], [ 2 ] ] )
gap> H:=RandomGraph(4);
Graph( Category := Graphs, Order := 4, Size := 8, Adjacencies :=
[ [ 3, 4 ], [ 4 ], [ 1, 2, 3, 4 ], [ 2 ] ] )
gap> Join(G,H);
Graph( Category := Graphs, Order := 4, Size := 8, Adjacencies :=
[ [ 3, 4 ], [ 4 ], [ 1, 2, 3, 4 ], [ 2 ] ] )
\endexample



$$
\xymatrix{
   {\bullet} \ar@{-}[d] & {\bullet} \ar@{-}[l] \\
   {\bullet} \ar@{-}[ur]
} \phantom{cm}
+ \phantom{cm}
\xymatrix{
   {\bullet} \\
   {\bullet} \ar@{-}[u]
}
=
\xymatrix{
   {\bullet} \ar@{-}[d] & {\bullet} \ar@{-}[l] \ar@{-}[r] & 
              {\bullet} \ar@{-}[lld] \ar@(ul,ur)@{-}[ll] \\
   {\bullet} \ar@{-}[ur] \ar@{-}[rr] & & 
              {\bullet} \ar@{-}[u] \ar@{-}[llu] \ar@{-}[lu]
} 
$$

\>GraphSum( <G>, <L> ) O

Returns the GraphSum of <G> and a list of graphs <L>.

The GraphSum is computed as follows:

Given <G> and a list of graphs $L = L_1, \dots L_n$,
if <G> is the trivial graph the result is $L_1.$ Otherwise we take
$g_1,g_2 \in G.$ If $g_1 \sim g_2$ we compute $L_1 + L_2$ and the
DisjointUnion in other case. We repeat this process for every
$g_i, g_{i+1}$ until $i+1 = n.$

If a graph is not given in a particular element of the list the
trivial graph will be used, <e.g.> $[H, , J, ] \equiv [H, T, J,
T]$ where <T> is the trivial graph. 

\beginexample
gap> G:=RandomGraph(4);
Graph( Category := Graphs, Order := 4, Size := 8, Adjacencies :=
[ [ 3, 4 ], [ 4 ], [ 1, 2, 3, 4 ], [ 2 ] ] )
gap> H:=RandomGraph(4);
Graph( Category := Graphs, Order := 4, Size := 8, Adjacencies :=
[ [ 3, 4 ], [ 4 ], [ 1, 2, 3, 4 ], [ 2 ] ] )
gap> GraphSum(G,H);
Graph( Category := Graphs, Order := 4, Size := 8, Adjacencies :=
[ [ 3, 4 ], [ 4 ], [ 1, 2, 3, 4 ], [ 2 ] ] )
\endexample


\>Composition( <G>, <H> ) O

Returns the composition of two graphs <G> and <H>, $G[H].$

A composition of graphs is obtained by calculating the GraphSum
of <G> with <H>, $G[H] = \hbox{GraphSum}(G, [H, \dots, H]).$

\beginexample
gap> G:=RandomGraph(4);
Graph( Category := Graphs, Order := 4, Size := 8, Adjacencies :=
[ [ 3, 4 ], [ 4 ], [ 1, 2, 3, 4 ], [ 2 ] ] )
gap> H:=RandomGraph(4);
Graph( Category := Graphs, Order := 4, Size := 8, Adjacencies :=
[ [ 3, 4 ], [ 4 ], [ 1, 2, 3, 4 ], [ 2 ] ] )
gap> Composition(G,H);
Graph( Category := Graphs, Order := 4, Size := 8, Adjacencies :=
[ [ 3, 4 ], [ 4 ], [ 1, 2, 3, 4 ], [ 2 ] ] )
\endexample



