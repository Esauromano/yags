% This file was created automatically from categories.msk.
% DO NOT EDIT!
\Chapter{Categories}

\Section{Graph Categories}
Using {\GAP} category facilities \YAGS \ defines a hierarchy of
graphs. The categories defined are as follows.

\>Graphs( ) C

Graphs are the base category used by \YAGS. This category contains
all graphs that can be represented in \YAGS. 



Among them we can find:

$$
\xymatrix{
   {\bullet} \ar@(dr,dl)[d] \ar[dr] \\
   {\bullet} \ar@(ul,ur)[u] \ar[r] & {\bullet}
} 
\hskip 1cm
\xymatrix{
   {\bullet} \ar@{-}[dr] & {\bullet} \ar[dl]\\
   {\bullet} & {\bullet} \ar@(ur,dr)[]\ar[l]
} 
$$
\bigskip
$$
\xymatrix{
   {\bullet} \ar@(ul,dl)[] \ar[dr] & & {\bullet} \ar@(l,d)[dl]\\
   & {\bullet} \ar@(r,u)[ur] & 
} 
\hskip 1cm
\xymatrix{
   & {\bullet} \ar@{-}[dl] \ar@{-}[dr] & & {\bullet} \ar@{-}[dl] \\
   {\bullet} & & {\bullet} \ar@{-}[dr] \\
   & {\bullet} \ar@{-}[ur] \ar@{-}[ul] & & {\bullet} \ar@{-}[uu] \\
}
$$

\>LooplessGraphs( ) C

Loopless Graphs are graphs which have no loops.



A loop is an arrow that starts and finishes on the same
vertex. 

$$
\xymatrix{
   {\bullet} \ar@(ur,dr)[]
} 
$$

Loopless graphs have no such arrows.

$$
\xymatrix{
   {\bullet} \ar[dr] & & {\bullet} \ar@(l,d)[dl]\\
   & {\bullet} \ar@(r,u)[ur] & 
} 
$$

\>UndirectedGraphs( ) C

Undirected Graphs are graphs which have no directed arrows.



Given two vertex $i,j$ in graph $G$ we will say that graph $G$ has an
*edge* $\{i,j\}$ if there is an arrow $(i,j)$ and and arrow $(j,i).$ 

$$
\xymatrix{
   {\bullet} \ar@(ur,dr)[r] & {\bullet} \ar@(dl,ul)[l]
}
\enskip
\equiv 
\enskip
\xymatrix{
   {\bullet} \ar@{-}[r] &  {\bullet}
}
$$

Undirected graphs have no arrows but only edges.

$$
\vcenter{\xymatrix{
   {\bullet} \ar@(ul,dl)[] \ar@(r,d)[dr] & & {\bullet} \ar@(l,d)[dl]\\
   & {\bullet} \ar@(r,u)[ur] \ar@(l,u)[ul] & 
}}
\enskip
\equiv
\hskip .5cm
\vcenter{\xymatrix{
   {\bullet} \ar@(ul,dl){-} \ar@{-}[dr] & & {\bullet} \ar@{-}[dl]\\
   & {\bullet} & 
}}
$$

\>OrientedGraphs( ) C

Oriented Graphs are graphs which have arrows in only one direction
between any two vertices. 



Oriented graphs have no edges but only arrows. 

$$
\xymatrix{
   {\bullet} \ar[dr] & & {\bullet} \ar[ll]\\
   & {\bullet} \ar[ur] & 
} 
$$

\>SimpleGraphs( ) C

Simple Graphs are graphs with no loops and undirected.


$$
\xymatrix{
   & {\bullet} \ar@{-}[dl] \ar@{-}[dr] & & {\bullet} \ar@{-}[dl] \\
   {\bullet} & & {\bullet} \ar@{-}[dr] \\
   & {\bullet} \ar@{-}[ur] \ar@{-}[ul] & & {\bullet} \ar@{-}[uu] \\
}
$$

The following figure shows the relationships among categories.

$$
\pstree[nodesep=5pt]{\Tr{Graphs}}
{
  \pstree{\Tr{Loopless}}
  {
         \Tr{Oriented}
         \Tr[name=S]{Simple Graphs}
  }
  \Tr[name=U]{Undirected}
}
\ncline[nodesep=5pt]{S}{U}
$$
\medskip\nobreak
\centerline{{\bf Figure 2:} Graph Categories}
\medskip

This relationship is important because when a graph is created it is
normalized to the category it belongs. For instance, if we create a
graph such as 
$$
\xymatrix{
   {\bullet} \ar[dr] & & {\bullet} \ar[ll]\\
   & {\bullet} \ar[ur] & 
}
$$
as a simple graph \YAGS\  will normalize the graph as
$$
\xymatrix{
   {\bullet} \ar@{-}[dr] & & {\bullet} \ar@{-}[ll]\\
   & {\bullet} \ar@{-}[ur] & 
}
$$
For further examples see the following section.


\Section{Default Category}

There are several ways to specify the category in which a new graph
will be created. There exists a <DefaultCategory> which tells \YAGS\  to
which category belongs any new graph by default. The <DefaultCategory> can be
changed using the following function.

\>SetDefaultGraphCategory( <C> ) F

Sets category C to be the default category for graphs. The default
category is used, for instance, when constructing new graphs. 

\beginexample
gap> SetDefaultGraphCategory(Graphs);
gap> g:=RandomGraph(4);
Graph( Category := Graphs, Order := 4, Size := 8, Adjacencies :=
[ [ 3, 4 ], [ 4 ], [ 1, 2, 3, 4 ], [ 2 ] ] )
\endexample

$$
\xymatrix{
   {\bullet} \ar@(dr,dl)[d] \ar[dr] & {\bullet} \ar[d]\\
   {\bullet} \ar@(ul,ur)[u] \ar[ur] & {\bullet} \ar[l] 
}
$$

RandomGraph creates a random graphs belonging to the category
graphs. The above graph has loops which are not permitted in
simple graphs.

\beginexample
gap> SetDefaultGraphCategory(SimpleGraphs);
gap> g:=CopyGraph(g);
Graph( Category := SimpleGraphs, Order := 4, Size := 5, Adjacencies :=
[ [ 3, 4 ], [ 3, 4 ], [ 1, 2, 4 ], [ 1, 2, 3 ] ] )
\endexample

Now G is a simple graph.

$$
\xymatrix{
   {\bullet} \ar[dr] & & {\bullet} \ar[ll]\\
   & {\bullet} \ar[ur] & 
}
$$




In order to handle graphs with different categories there two
functions available.

\>GraphCategory( [<G>, ... ] ) F

Returns the minimal common category to a list of graphs. See
Section "Categories" for the relationship among categories.
 
If the list is empty the default category is returned. 



\>TargetGraphCategory( [<G>, ... ] ) F

Returns the category which will be used to process a list of
graphs. If an option category has been given it will return that
category. Otherwise it will behave as Function <GraphCategory>
("GraphCategory"). See Section "Categories" for the relationship
among categories. 



Finally we can test if a single graph belongs to a given category.

\>in( <G>, <C> ) O

Returns `true' if graph <G> belongs to category <C> and `false' otherwise.






