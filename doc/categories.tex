\Chapter{Categories}

\Section{Graph Categories}
Using {\GAP} category facilities {\YAGS} defines a hierarchy of
graphs. The categories defined are as follows.

{\bf Declaration of:  }{Graphs}

Among them we can find:

$$
\xymatrix{
   {\bullet} \ar@(dr,dl)[d] \ar[dr] \\
   {\bullet} \ar@(ul,ur)[u] \ar[r] & {\bullet}
} 
\hskip 1cm
\xymatrix{
   {\bullet} \ar@{-}[dr] & {\bullet} \ar[dl]\\
   {\bullet} & {\bullet} \ar@(ur,dr)[]\ar[l]
} 
$$
\bigskip
$$
\xymatrix{
   {\bullet} \ar@(ul,dl)[] \ar[dr] & & {\bullet} \ar@(l,d)[dl]\\
   & {\bullet} \ar@(r,u)[ur] & 
} 
\hskip 1cm
\xymatrix{
   & {\bullet} \ar@{-}[dl] \ar@{-}[dr] & & {\bullet} \ar@{-}[dl] \\
   {\bullet} & & {\bullet} \ar@{-}[dr] \\
   & {\bullet} \ar@{-}[ur] \ar@{-}[ul] & & {\bullet} \ar@{-}[uu] \\
}
$$

{\bf Declaration of:  }{LooplessGraphs}

A loop is an arrow that starts and finishes on the same
vertex. 

$$
\xymatrix{
   {\bullet} \ar@(ur,dr)[]
} 
$$

Loopless graphs have no such arrows.

$$
\xymatrix{
   {\bullet} \ar[dr] & & {\bullet} \ar@(l,d)[dl]\\
   & {\bullet} \ar@(r,u)[ur] & 
} 
$$

{\bf Declaration of:  }{UndirectedGraphs}

Given two vertex $i,j$ in graph $G$ we will say that graph $G$ has an
*edge* $\{i,j\}$ if there is an arrow $(i,j)$ and and arrow $(j,i).$ 

$$
\xymatrix{
   {\bullet} \ar@(ur,dr)[r] & {\bullet} \ar@(dl,ul)[l]
}
\enskip
\equiv 
\enskip
\xymatrix{
   {\bullet} \ar@{-}[r] &  {\bullet}
}
$$

Undirected graphs have no arrows but only edges.

$$
\vcenter{\xymatrix{
   {\bullet} \ar@(ul,dl)[] \ar@(r,d)[dr] & & {\bullet} \ar@(l,d)[dl]\\
   & {\bullet} \ar@(r,u)[ur] \ar@(l,u)[ul] & 
}}
\enskip
\equiv
\hskip .5cm
\vcenter{\xymatrix{
   {\bullet} \ar@(ul,dl){-} \ar@{-}[dr] & & {\bullet} \ar@{-}[dl]\\
   & {\bullet} & 
}}
$$

{\bf Declaration of:  }{OrientedGraphs}

Oriented graphs have no edges but only arrows. 

$$
\xymatrix{
   {\bullet} \ar[dr] & & {\bullet} \ar[ll]\\
   & {\bullet} \ar[ur] & 
} 
$$

{\bf Declaration of:  }{SimpleGraphs}
$$
\xymatrix{
   & {\bullet} \ar@{-}[dl] \ar@{-}[dr] & & {\bullet} \ar@{-}[dl] \\
   {\bullet} & & {\bullet} \ar@{-}[dr] \\
   & {\bullet} \ar@{-}[ur] \ar@{-}[ul] & & {\bullet} \ar@{-}[uu] \\
}
$$

The following figure shows the relationships among categories.

$$
\pstree[nodesep=5pt]{\Tr{Graphs}}
{
  \pstree{\Tr{Loopless}}
  {
         \Tr{Oriented}
         \Tr[name=S]{Simple Graphs}
  }
  \Tr[name=U]{Undirected}
}
\ncline[nodesep=5pt]{S}{U}
$$
\medskip\nobreak
\centerline{{\bf Figure 2:} Graph Categories}
\medskip

This relationship is important because when a graph is created it is
normalized to the category it belongs. For instance, if we create a
graph such as 
$$
\xymatrix{
   {\bullet} \ar[dr] & & {\bullet} \ar[ll]\\
   & {\bullet} \ar[ur] & 
}
$$
as a simple graph {\YAGS}  will normalize the graph as
$$
\xymatrix{
   {\bullet} \ar@{-}[dr] & & {\bullet} \ar@{-}[ll]\\
   & {\bullet} \ar@{-}[ur] & 
}
$$
For further examples see the following section.


\Section{Default Category}

There are several ways to specify the category in which a new graph
will be created. There exists a <DefaultCategory> which tells {\YAGS}  to
which category belongs any new graph by default. The <DefaultCategory> can be
changed using the following function.

{\bf Declaration of:  }{SetDefaultGraphCategory}

In order to handle graphs with different categories there two
functions available.

{\bf Declaration of:  }{GraphCategory}

{\bf Declaration of:  }{TargetGraphCategory}

Finally we can test if a single graph belongs to a given category.

{\bf Declaration of:  }{in}




