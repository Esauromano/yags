% This file was created automatically from constructing.msk.
% DO NOT EDIT!
\Chapter{Constructing graphs}

\Section{Primitives}

The following functions create new graphs from a variety of sources.

\>Graph( <R> ) O

Creates a graph from the record <R>. There are two representations
from which a graph can be created:
\beginlist
\item{1.}
From an adjacency list.
\item{2.}
From an adjacencies matrix.
\endlist
The record must, therefore, provide a field <Adjacencies> containing
the list of adjacencies for each vertex or alternatively a field
<AdjMatrix> with the adjacency matrix. Additionaly we must provide
the category, or categories, to which the new graph belongs. 


\>IntersectionGraph( <L> ) F

Creates a graph from list <L> where <L> is an intersection list.



The following functions create graphs from existing graphs

\>CopyGraph( <G> ) O

Creates a fresh copy of graph <G>.


\>InducedSubgraph( <G>, <V> ) O

Creates an induced graph from graph <G> using only vertices <V>.


\>RemoveVertices( <G>, <V> ) O

Creates a graph from graph <G> removing vertices <V>.


\>AddEdges( <G>, <E> ) O

Creates a graph from graph <G> adding the set of edges <E>.


\>RemoveEdges( <G>, <E> ) O

Creates a graph from graph <G> removing edges <E>.



\>CliqueGraph( <G> ) A
\>CliqueGraph( <G>, <m> ) O

The clique graph of graph <G>, $K(G)$. The additional parameter <m>
stops the operation when a maximum of <m> cliques have been found.

\beginexample
gap> G:=SunGraph(4);
Graph( Category := SimpleGraphs, Order := 8, Size := 14, Adjacencies :=
[ [ 2, 8 ], [ 1, 3, 4, 6, 8 ], [ 2, 4 ], [ 2, 3, 5, 6, 8 ], [ 4, 6 ],
  [ 2, 4, 5, 7, 8 ], [ 6, 8 ], [ 1, 2, 4, 6, 7 ] ] )
gap> CliqueGraph(G);
Graph( Category := SimpleGraphs, Order := 5, Size := 8, Adjacencies :=
[ [ 2, 3, 4, 5 ], [ 1, 3, 4 ], [ 1, 2, 5 ], [ 1, 2, 5 ], [ 1, 3, 4 ] ] )
\endexample



\Section{Families}

The following functions return well known graphs. Most of them can be
found in Brandstadt, Le and Spinrad.

\>DiscreteGraph( <n> ) F

Returns a Discrete Graph of order <n>. A discrete graph is a graph
where vertices are unconnected.

\beginexample
gap> DiscreteGraph(4);
Graph( Category := SimpleGraphs, Order := 4, Size := 0, Adjacencies :=
[ [  ], [  ], [  ], [  ] ] )
\endexample



4-Discrete Graph
\enskip
$\vcenter{\xymatrix{
   {\bullet} & {\bullet} \\
   {\bullet} & {\bullet} 
}}$

\>CompleteGraph( <n> ) F

Returns a Complete Graph of order <n>. A complete graph is a graph
where all vertices are connected to each other.

\beginexample
gap> CompleteGraph(4);
Graph( Category := SimpleGraphs, Order := 4, Size := 6, Adjacencies :=
[ [ 2, 3, 4 ], [ 1, 3, 4 ], [ 1, 2, 4 ], [ 1, 2, 3 ] ] )
\endexample



4-Complete Graph
\enskip
$\vcenter{\xymatrix{
   {\bullet} \ar@{-}[r] \ar@{-}[d] & {\bullet} \ar@{-}[d] \ar@{-}[dl]\\
   {\bullet} \ar@{-}[r] & {\bullet} \ar@{-}[ul]
}}$

\>PathGraph( <n> ) F

Returns a Path Graph of order <n>. A path graph is a graph
connected forming a path.

\beginexample
gap> PathGraph(4);
Graph( Category := SimpleGraphs, Order := 4, Size := 3, Adjacencies :=
[ [ 2 ], [ 1, 3 ], [ 2, 4 ], [ 3 ] ] )
\endexample



4-Path Graph
\enskip
$\vcenter{\xymatrix{
   {\bullet} \ar@{-}[r] & {\bullet} \ar@{-}[r] &
   {\bullet} \ar@{-}[r] & {\bullet} 
}}$

\>CycleGraph( <n> ) F

Returns a Cycle Graph of order <n>. A cycle graph is a path graph
where the vertices at the ends are connected.

\beginexample
gap> CycleGraph(5);
Graph( Category := SimpleGraphs, Order := 5, Size := 5, Adjacencies :=
[ [ 2, 5 ], [ 1, 3 ], [ 2, 4 ], [ 3, 5 ], [ 1, 4 ] ] )
\endexample


5-Cycle Graph
\enskip
\xy /r10mm/:
\xypolygon5{@{*}}
\endxy

\>CubeGraph( <n> ) F

Returns a Cube Graph of order <n>. A cube graph is a graph where
each vertex has degree <n>.

\beginexample
gap> CubeGraph(3);
Graph( Category := SimpleGraphs, Order := 8, Size := 12, Adjacencies :=
[ [ 2, 3, 5 ], [ 1, 4, 6 ], [ 1, 4, 7 ], [ 2, 3, 8 ], [ 1, 6, 7 ],
[ 2, 5, 8 ], [ 3, 5, 8 ], [ 4, 6, 7 ] ] )
\endexample


3-Cube Graph
\enskip
\objectmargin={1pt}
\xygraph{
 \bullet-[r]\bullet-[d]\bullet-[l]\bullet-[u],
  -[ur(.5)]\bullet-[r]\bullet-[d]\bullet-[l]\bullet-[u],
  [r]-[ur(.5)],
  [r][d]-[ur(.5)],
  [r][d][l]-[ur(.5)],
}

\>OctahedralGraph( <n> ) F

\beginexample
gap> OctahedralGraph(3);
Graph( Category := SimpleGraphs, Order := 6, Size := 12, Adjacencies :=
[ [ 3, 4, 5, 6 ], [ 3, 4, 5, 6 ], [ 1, 2, 5, 6 ], [ 1, 2, 5, 6 ],
[ 1, 2, 3, 4 ], [ 1, 2, 3, 4 ] ] )
\endexample


3-Octahedral Graph
\enskip
\objectmargin={1pt}
\xygraph{
 \bullet
   ([u][r(.5)][ur(.25)]\bullet-?,
    [d][r(.5)][ur(.25)]\bullet-?,
    -[r]\bullet,
    -[ur(.5)]\bullet,
   )
 [r]
   ([u][l(.5)][ur(.25)]-?,
    [d][l(.5)][ur(.25)]-?,
   )
 -[ur(.5)]\bullet
   ([u][l(.5)][dl(.25)]-?,
    [d][l(.5)][dl(.25)]-?,
   )
 -[l]
   ([u][r(.5)][dl(.25)]-?,
    [d][r(.5)][dl(.25)]-?,
   )
}

\>JohnsonGraph( <n>, <r> ) F

Returns a Johnson Graph $J(n,r).$ A Johnson Graph is a 
graph constructed as follows. Each vertex represents a subset of
the set $\{1,\dots,n\}$ with cardinality $r.$ $$V(J(n,r)) = \{ X \subset
\{1,\dots,n\} | |X| = r \}$$ and there is an edge between two
vertices if and only if the cardinality of the intersection of the
sets they represent is $r-1$ $$X \sim X' \hbox{ iff } |X \cup X'| = r-1.$$

\beginexample
gap> JohnsonGraph(4,2);
Graph( Category := SimpleGraphs, Order := 6, Size := 12, Adjacencies :=
[ [ 2, 3, 4, 5 ], [ 1, 3, 4, 6 ], [ 1, 2, 5, 6 ], [ 1, 2, 5, 6 ],
 [ 1, 3, 4, 6 ], [ 2, 3, 4, 5 ] ] )
\endexample


\vskip .5cm
4,2-Johnson Graph
\enskip
$\vcenter{\xymatrix{
   {\bullet} \ar@{-}[r] \ar@{-}[d] \ar@{-}[dr]
     & {\bullet} \ar@{-}[r] \ar@{-}[dr] \ar@{-}[dl]
       & {\bullet} \ar@{-}[d] \ar@{-}[dl] \ar@(ul,ur)@{-}[ll] \\
   {\bullet} \ar@{-}[r]
     & {\bullet} \ar@{-}[r]
       & {\bullet} \ar@(dl,dr)@{-}[ll]
}}$

\phantom{J}

\>CompleteBipartiteGraph( <n>, <m> ) F

Returns a Complete Bipartite Graph of order <n> + <m>. A complete
bipartite graph is the result of joining two Discrete graphs and
adding edges to connect all vertices of each graph.

\beginexample
gap> CompleteBipartiteGraph(2,3);
Graph( Category := SimpleGraphs, Order := 5, Size := 6, Adjacencies :=
[ [ 3, 4, 5 ], [ 3, 4, 5 ], [ 1, 2 ], [ 1, 2 ], [ 1, 2 ] ] )
\endexample


2,3-Complete Bipartite Graph
\enskip
$\vcenter{
\xygraph{
 \bullet[r][u(.5)]\bullet[d]\bullet[d]\bullet
   [l][u(.5)]\bullet,
 ([r][u(.5)]-?,
  [r][u(.5)][d]-?,
  [r][u(.5)][d][d]-?,
 ),
 [d]
  ([r][u(1.5)]-?,
   [r][u(.5)]-?,
   [r][d(.5)]-?,
  )
}
}$

\>CompleteMultipartiteGraph( <n1>, <n2> [, <n3> ...] ) F

Returns a Complete Multipartite Graph of order <n1> + <n2> +
\dots. A complete multipartite graph is the result of joining
Discrete graphs and adding edges to connect all vertices of each
graph. 

\beginexample
gap> CompleteMultipartiteGraph(2,2,2);
Graph( Category := SimpleGraphs, Order := 6, Size := 12, Adjacencies :=
[ [ 3, 4, 5, 6 ], [ 3, 4, 5, 6 ], [ 1, 2, 5, 6 ], [ 1, 2, 5, 6 ],
 [ 1, 2, 3, 4 ], [ 1, 2, 3, 4 ] ] )
\endexample


2,2,2-Complete Multipartite Graph
\enskip
$\vcenter{
\xygraph{
 \bullet[d(.5)]\bullet[rd(.5)]\bullet[r(.5)]\bullet[ru(.5)]
   \bullet[u(.5)]\bullet,
 ([d(.5)][rd(.5)]-?,
  [d(.5)][rd(.5)][r(.5)]-?,
  [d(.5)][rd(.5)][r(.5)][ru(.5)]-?,
  [d(.5)][rd(.5)][r(.5)][ru(.5)][u(.5)]-?,
 ),
 [d(.5)]
  ([rd(.5)]-?,
   [rd(.5)][r(.5)]-?,
   [rd(.5)][r(.5)][ru(.5)]-?,
   [rd(.5)][r(.5)][ru(.5)][u(.5)]-?,
  ),
 [d(.5)][rd(.5)]
  ([r(.5)][ru(.5)]-?,
   [r(.5)][ru(.5)][u(.5)]-?,
  ),
 [d(.5)][rd(.5)][r(.5)]
  ([ru(.5)]-?,
   [ru(.5)][u(.5)]-?,
  )
}
}$

\>RandomGraph( <n>, <p> ) F
\>RandomGraph( <n> ) F

Returns a Random Graph of order <n>. The first form additionally
takes a parameter <p>, the probability of an edge to exist. A
probability 1 will return a Complete Graph and a probability 0 a
Discrete Graph.

\beginexample
gap> RandomGraph(5);
Graph( Category := SimpleGraphs, Order := 5, Size := 5, Adjacencies :=
[ [ 4, 5 ], [ 4, 5 ], [  ], [ 1, 2, 5 ], [ 1, 2, 4 ] ] )
\endexample


5-Random Graph

\>WheelGraph( <N> ) O
\>WheelGraph( <N>, <Radius> ) O

\beginexample
WheelGraph(5);
gap> Graph( Category := SimpleGraphs, Order := 6, Size := 10, Adjacencies :=
[ [ 2, 3, 4, 5, 6 ], [ 1, 3, 6 ], [ 1, 2, 4 ], [ 1, 3, 5 ], [ 1, 4, 6 ],
 [ 1, 2, 5 ] ] )
\endexample

#FIXME: Poner un ejemplo con radio.


\undogap
Wheel Graph of Order 5
\enskip
\xy /r10mm/:,
\drop{\bullet}\xypolygon5{~<{-}\bullet}
\endxy
\dogap

\>FanGraph( <N> ) F

\beginexample
gap> FanGraph(4);
Graph( Category := SimpleGraphs, Order := 6, Size := 9, Adjacencies :=
[ [ 2, 3, 4, 5, 6 ], [ 1, 3 ], [ 1, 2, 4 ], [ 1, 3, 5 ], [ 1, 4, 6 ],
[ 1, 5 ] ] )
\endexample


4-Fan Graph
\enskip
$\vcenter{
\objectmargin={1pt}
\xygraph{
 \bullet (-[u]\bullet-[r(.6)]-[dr(.4)]-?
   ,[u][r(.6)]\bullet-?
   ,-[r]\bullet-[u(.6)]\bullet-?
}
}$

\>SunGraph( <N> ) F

\beginexample
gap> SunGraph(4);
Graph( Category := SimpleGraphs, Order := 8, Size := 14, Adjacencies :=
[ [ 2, 8 ], [ 1, 3, 4, 6, 8 ], [ 2, 4 ], [ 2, 3, 5, 6, 8 ], [ 4, 6 ],
[ 2, 4, 5, 7, 8 ], [ 6, 8 ], [ 1, 2, 4, 6, 7 ] ] )
\endexample


4-Sun Graph
\enskip
$\vcenter{
\objectmargin={1pt}
\xygraph{
 \bullet-[dr]\bullet-[dl]\bullet-[ul]\bullet-[ur],
 [dr(.5)]\bullet-[d]\bullet-[l]\bullet-[u]\bullet-[r],
 [dr(.5)]-[dl][r]-[ul]
}
}$

\>SpikyGraph( <N> ) F

\beginexample
gap> SpikyGraph(3);
Graph( Category := SimpleGraphs, Order := 6, Size := 9, Adjacencies :=
[ [ 2, 3, 4, 5 ], [ 1, 3, 4, 6 ], [ 1, 2, 5, 6 ], [ 1, 2 ], [ 1, 3 ],
[ 2, 3 ] ] )
\endexample


3-Spiky Graph
\enskip
$\vcenter{
\objectmargin={1pt}
\xygraph{
 \bullet,-[r(.5)]\bullet-[dr(.5)]\bullet-[u(.5)]\bullet-[l(.5)]\bullet
  -[dl(.5)]\bullet-[r],
  -[d(.5)]-[dr(.5)]\bullet-[ur(.5)]
}
}$
\>`TrivialGraph' V

\beginexample
gap> TrivialGraph;
Graph( Category := SimpleGraphs, Order := 1, Size := 0, Adjacencies :=
[ [  ] ] )
\endexample


Trivial Graph
\xygraph{
 \bullet
}


\>`DiamondGraph' V

\beginexample
gap> DiamondGraph;
Graph( Category := SimpleGraphs, Order := 4, Size := 5, Adjacencies :=
[ [ 2, 3, 4 ], [ 1, 3 ], [ 1, 2, 4 ], [ 1, 3 ] ] )
\endexample


Diamond Graph
\enskip
$\vcenter{
\objectmargin={1pt}
\xygraph{
 \bullet,-[dl(.5)]\bullet-[dr(.5)]\bullet-[ur(.5)]\bullet-[l]\bullet,
  -[dr(.5)]\bullet
}
}$

\>`ClawGraph' V

\beginexample
gap> ClawGraph;
Graph( Category := SimpleGraphs, Order := 4, Size := 3, Adjacencies :=
[ [ 2, 3, 4 ], [ 1 ], [ 1 ], [ 1 ] ] )
\endexample


Claw Graph
\objectmargin={1pt}
\xygraph{
 \bullet,-[u(.5)]\bullet,-[dl(.5)]\bullet,-[dr(.5)]\bullet
}

\>`PawGraph' V

\beginexample
gap> PawGraph;
Graph( Category := SimpleGraphs, Order := 4, Size := 4, Adjacencies :=
[ [ 2 ], [ 1, 3, 4 ], [ 2, 4 ], [ 2, 3 ] ] )
\endexample


Paw Graph
\enskip
$\vcenter{
\objectmargin={1pt}
\xygraph{
 \bullet,-[d(.5)]\bullet-[dl(.5)]\bullet-[r]\bullet-[ul(.5)]
}
}$

\>`HouseGraph' V

\beginexample
gap> HouseGraph;
Graph( Category := SimpleGraphs, Order := 5, Size := 6, Adjacencies :=
[ [ 2, 3 ], [ 1, 3, 4 ], [ 1, 2, 5 ], [ 2, 5 ], [ 3, 4 ] ] )
\endexample


House Graph
\enskip
$\vcenter{
\objectmargin={1pt}
\xygraph{
 \bullet,-[dl(.5)]\bullet-[d(.5)]\bullet-[r]\bullet-[u(.5)]\bullet-[l]\bullet,
  -[dr(.5)]\bullet
}
}$

\>`BullGraph' V


Bull Graph
\enskip
$\vcenter{
\objectmargin={1pt}
\xygraph{
 \bullet,-[ul(.5)]\bullet-[u(.5)]\bullet,
  -[ur(.5)]\bullet-[u(.5)]\bullet,
  [ur(.5)]-[l]\bullet
}
}$

\>`AntennaGraph' V


Antenna Graph
\enskip
$\vcenter{
\objectmargin={1pt}
\xygraph{
 \bullet,-[dl(.5)]\bullet-[d(.5)]\bullet-[r]\bullet-[u(.5)]\bullet-[l]\bullet,
  -[dr(.5)]\bullet,
  -[u(.5)]\bullet
}
}$

\>`KiteGraph' V


Kite Graph
\objectmargin={1pt}
\xygraph{
 \bullet,-[l(.5)]\bullet,
 -[dr(.5)]\bullet-[ur(.5)]\bullet-[ul(.5)]\bullet-[dl(.5)],
 [dr(.5)]-[u]
}

\>`Tetrahedron' V

\beginexample
gap> Tetrahedron;
Graph( Category := SimpleGraphs, Order := 4, Size := 6, Adjacencies :=
[ [ 2, 3, 4 ], [ 1, 3, 4 ], [ 1, 2, 4 ], [ 1, 2, 3 ] ] )
\endexample


Tetrahedron
\xygraph{ %/r10mm/:
 \bullet (-[u(.5)]\bullet,-[dr(.5)]\bullet,-[dl(.5)]\bullet),
 [u(.5)] ([d(.5)][dr(.5)]-?, [d(.5)][dl(.5)]-?),
 [dr(.5)]-[l]
}


\>`Octahedron' V

\beginexample
gap> Octahedron;
Graph( Category := SimpleGraphs, Order := 6, Size := 12, Adjacencies :=
[ [ 3, 4, 5, 6 ], [ 3, 4, 5, 6 ], [ 1, 2, 5, 6 ], [ 1, 2, 5, 6 ],
[ 1, 2, 3, 4 ], [ 1, 2, 3, 4 ] ] )
\endexample


\undogap
Octahedron
\enskip
$\vcenter{
\xy /r2pc/:
{\xypolygon3"A"{~:{(0,.7)::}\bullet}},+(.7,1.1),
{\xypolygon3"B"{~:{(-.85,0):(-.150,.8)::}\bullet}},
{"A1"\PATH~={**@{-}}'"B2"'"A3"'"B1"'"A2"'"B3"'"A1"}
\endxy
}$
\dogap

\>`Cube' V

\beginexample
gap> Cube;
Graph( Category := SimpleGraphs, Order := 8, Size := 12, Adjacencies :=
[ [ 2, 3, 5 ], [ 1, 4, 6 ], [ 1, 4, 7 ], [ 2, 3, 8 ], [ 1, 6, 7 ],
[ 2, 5, 8 ], [ 3, 5, 8 ], [ 4, 6, 7 ] ] )
\endexample


Cube Graph
\enskip
$\vcenter{
\objectmargin={1pt}
\xygraph{
 \bullet-[r]\bullet-[d]\bullet-[l]\bullet-[u],
  -[ur(.5)]\bullet-[r]\bullet-[d]\bullet-[l]\bullet-[u],
  [r]-[ur(.5)],
  [r][d]-[ur(.5)],
  [r][d][l]-[ur(.5)],
}
}$

\>`Icosahedron' V

\beginexample
gap> Icosahedron;
Graph( Category := SimpleGraphs, Order := 12, Size := 30, Adjacencies :=
[ [ 2, 3, 4, 5, 6 ], [ 1, 3, 6, 9, 10 ], [ 1, 2, 4, 10, 11 ],
 [ 1, 3, 5, 7, 11 ], [ 1, 4, 6, 7, 8 ], [ 1, 2, 5, 8, 9 ],
 [ 4, 5, 8, 11, 12 ], [ 5, 6, 7, 9, 12 ], [ 2, 6, 8, 10, 12 ],
 [ 2, 3, 9, 11, 12 ], [ 3, 4, 7, 10, 12 ], [ 7, 8, 9, 10, 11 ] ] )
\endexample


\undogap
Icosahedron
\xy /r3pc/:
 ="S",
 +(0,1.5)*{\bullet}="A",
 "S",*{\bullet},
 {\xypolygon5"B"{~:{(1.5,0):(0,.33)::}~<>{;"A"**@{-}}\bullet}},
 {\xypolygon5"C"{~={55}~:{(.7,0):(0,.33)::}~<{-}\bullet}},
 {"B1"\PATH~={**@{-}}'"C1"},{"B1"\PATH~={**@{-}}'"C5"},
 {"B2"\PATH~={**@{-}}'"C1"},{"B2"\PATH~={**@{-}}'"C2"},
 {"B3"\PATH~={**@{-}}'"C2"},{"B3"\PATH~={**@{-}}'"C3"},
 {"B4"\PATH~={**@{-}}'"C3"},{"B4"\PATH~={**@{-}}'"C4"},
 {"B5"\PATH~={**@{-}}'"C4"},{"B5"\PATH~={**@{-}}'"C5"}
\endxy
\dogap

\>`Dodecahedron' V


\undogap
Dodecahedron
\enskip
\xy /l1.5pc/:,
  {\xypolygon5"A"{\bullet}},
  {\xypolygon5"B"{~:{(1.875,0):}~>{}\bullet}},
  {\xypolygon5"C"{~:{(-2.95,0):}~>{}\bullet}},
  {\xypolygon5"D"{~:{(-3.75,0):}\bullet}},
  {"A1"\PATH~={**@{-}}'"B1"'"C4"'"B2"},
  {"A2"\PATH~={**@{-}}'"B2"'"C5"'"B3"},
  {"A3"\PATH~={**@{-}}'"B3"'"C1"'"B4"},
  {"A4"\PATH~={**@{-}}'"B4"'"C2"'"B5"},
  {"A5"\PATH~={**@{-}}'"B5"'"C3"'"B1"},
  "C1";"D1"**@{-},"C2";"D2"**@{-},
  "C3";"D3"**@{-},"C4";"D4"**@{-},
  "C5";"D5"**@{-}
\endxy

%ar@{-}[r] \ar@{-}[dar@{-}[r] \ar@{-}[d]
%ar@{-}[r] \ar@{-}[d]

%\xymatrix{
%   {\bullet}  & {\bullet} \ar@{-}[d]\\
%   {\bullet} \ar@{-}[r] & {\bullet} 
%}  

\Section{Unary operations}

These are operations that can be performed over graphs.

\>LineGraph( <G> ) O

Returns the line graph of graph <G>. The line graph is the
intersection graph of the edges of <G>, <i.e.> the vertices of
$L(G)$ are the edges of <G> two of them being adjacent iff they
are incident. 

\beginexample 
gap> G:=Tetrahedron;
Graph( Category := SimpleGraphs, Order := 4, Size := 6, Adjacencies :=
[ [ 2, 3, 4 ], [ 1, 3, 4 ], [ 1, 2, 4 ], [ 1, 2, 3 ] ] )
gap> LineGraph(G);
Graph( Category := SimpleGraphs, Order := 6, Size := 12, Adjacencies :=
[ [ 2, 3, 4, 5 ], [ 1, 3, 4, 6 ], [ 1, 2, 5, 6 ], [ 1, 2, 5, 6 ],
  [ 1, 3, 4, 6 ], [ 2, 3, 4, 5 ] ] )
\endexample


LineGraph(\xygraph{ %/r10mm/:
 \bullet (-[u(.5)]\bullet,-[dr(.5)]\bullet,-[dl(.5)]\bullet),
 [u(.5)] ([d(.5)][dr(.5)]-?, [d(.5)][dl(.5)]-?),
 [dr(.5)]-[l]
}) = 
\xygraph{[u(.5)]
 \bullet-[r]\bullet-[dl(.5)]\bullet-[dl(.5)]\bullet
   -[ul(.5)]\bullet-[ul(.5)]\bullet-[r]-[dl(.5)]-[r]-[ul(.5)],
 [r(.5)][ur(.5)] ([ll]-@(ur,ul)?, [dl]-@(r,d)?),
 [l(.5)][ul(.5)] ([dr]-@(l,d)?)
}

\>ComplementGraph( <G> ) A

Computes the complement of graph <G>. The complement of a graph is
created as follows:
Create a graph <G'> with same vertices of <G>. For each <x>, <y>
$\in$ <G> if <x> $\nsim$ <y> in <G> then <x> $\sim$ <y> in <G'>

\beginexample 
gap> G:=ClawGraph;
Graph( Category := SimpleGraphs, Order := 4, Size := 3, Adjacencies :=
[ [ 2, 3, 4 ], [ 1 ], [ 1 ], [ 1 ] ] )
gap> ComplementGraph(G);
Graph( Category := SimpleGraphs, Order := 4, Size := 3, Adjacencies :=
[ [  ], [ 3, 4 ], [ 2, 4 ], [ 2, 3 ] ] )
\endexample


ComplementGraph$(\vcenter{\xymatrix{
   {\bullet} \ar@{-}[r] \ar@{-}[d] & {\bullet} \ar@{-}[d]\\
   {\bullet} \ar@{-}[r] & {\bullet} 
}}
) = 
\vcenter{\xymatrix{
   {\bullet} \ar@{-}[rd] & {\bullet} \ar@{-}[dl]\\
   {\bullet} & {\bullet} 
}}$

\>QuotientGraph( <G>, <P> ) O
\>QuotientGraph( <G>, <L1>, <L2> ) O

Returns the quotient graph of graph <G> given a vertex partition
<P>, by identifying any two vertices in the same part. 
The vertices of the quotient
graph are the parts in the partition <P> two of them
being adjacent iff any two of the vertices in the parts are
adjacent in <G>. Singletons may be omited in P.

In its second form, QuotientGraph identifies each vertex in list L1, 
with the corresponding vertex in list L2. L1 and L2 must have the same length,
but any or both of them may have repetitions.

\beginexample 
gap> G:=HouseGraph;
Graph( Category := SimpleGraphs, Order := 5, Size := 6, Adjacencies :=
[ [ 2, 3 ], [ 1, 3, 4 ], [ 1, 2, 5 ], [ 2, 5 ], [ 3, 4 ] ] )
gap> QuotientGraph(G,[[1,2],[4,5]]);
Graph( Category := SimpleGraphs, Order := 3, Size := 3, Adjacencies :=
[ [ 2, 3 ], [ 1, 3 ], [ 1, 2 ] ] )
\endexample


\undogap
\objectmargin={1pt}
QuotientGraph$(\vcenter{\xygraph{
 \bullet([ur(.1)]*{\scriptstyle 1})
  -[dl(.5)]\bullet([l(.1)]*{\scriptstyle 2})
  -[d(.5)]\bullet([l(.1)]*{\scriptstyle 3})
  -[r]\bullet([r(.1)]*{\scriptstyle 4})
  -[u(.5)]\bullet([r(.1)]*{\scriptstyle 5})
   ([l]\bullet-?
    [ul(.5)]\bullet-?
   )
}}, [2,3],[4,5]) =
\vcenter{\xygraph{
 [u(.2)]\bullet([ur(.1)]*{\scriptstyle 1})
 -[dr(.5)]\bullet([r(.1)]*{\scriptstyle 3})
 -[l]\bullet([l(.1)]*{\scriptstyle 2})
 ([ur(.5)]\bullet-?)
}}$
\dogap

\Section{Binary operations}

These are binary operations that can be performed over graphs.

\>BoxProduct( <G>, <H> ) O

Returns the box product of two graphs <G> and <H> (also called
cartesian product), <G> $\square$ <H>.

The box product is calculated as follows:

For each pair of vertices $g \in G, h \in H$ we create a vertex
$(g,h).$ Given two such vertices $(g,h)$ and $(g',h')$ they are
adjacent <iff> $g = g'$ and $h \sim h'$ or $g \sim g'$ and $h = h'.$

\beginexample
gap> G:=GraphByAdjMatrix([[false,true],[true,false]]);
Graph( Category := SimpleGraphs, Order := 2, Size := 1, Adjacencies :=
[ [ 2 ], [ 1 ] ] )
gap> BoxProduct(G,G);
Graph( Category := SimpleGraphs, Order := 4, Size := 4, Adjacencies :=
[ [ 2, 3 ], [ 1, 4 ], [ 1, 4 ], [ 2, 3 ] ] )
\endexample



$$
\vcenter{\xymatrix{
   {\bullet} \\
   {\bullet} \ar@{-}[u]
}}
\square
\vcenter{\xymatrix{
   {\bullet} \\
   {\bullet} \ar@{-}[u]
}}
=
\vcenter{\xymatrix{
   {\bullet} \ar@{-}[r] \ar@{-}[d] & {\bullet} \ar@{-}[d]\\
   {\bullet} \ar@{-}[r] & {\bullet} 
}}
$$

\>TimesProduct( <G>, <H> ) O

Returns the times product of two graphs <G> and <H>, <G> $\times$ <H>.

The times product is computed as follows:

For each pair of vertices $g \in G, h \in H$ we create a vertex
$(g,h).$ Given two such vertices $(g,h)$ and $(g',h')$ they are
adjacent <iff> $g \sim g'$ and $h \sim h'.$

\beginexample
gap> G:=GraphByAdjMatrix([[false,true],[true,false]]);
Graph( Category := SimpleGraphs, Order := 2, Size := 1, Adjacencies :=
[ [ 2 ], [ 1 ] ] )
gap> TimesProduct(G,G);
Graph( Category := SimpleGraphs, Order := 4, Size := 2, Adjacencies :=
[ [ 4 ], [ 3 ], [ 2 ], [ 1 ] ] )
\endexample



$$
\vcenter{\xymatrix{
   {\bullet} \\
   {\bullet} \ar@{-}[u]
}}
\times
\vcenter{\xymatrix{
   {\bullet} \\
   {\bullet} \ar@{-}[u]
}}
=
\vcenter{\xymatrix{
   {\bullet} \ar@{-}[rd] & {\bullet} \ar@{-}[dl]\\
   {\bullet} & {\bullet} 
}}
$$

\>BoxTimesProduct( <G>, <H> ) O

Returns the box times product of two graphs <G> and <H>, <G>
$\boxtimes$ <H>. 

The box times product is calculated as follows:

For each pair of vertices $g \in G, h \in H$ we create a vertex
$(g,h).$ Given two such vertices $(g,h)$ and $(g',h')$ such that
$(g,h) \neq (g',h')$ they are adjacent <iff> $g \simeq g'$ and $h
\simeq h'.$ 

\beginexample
gap> G:=GraphByAdjMatrix([[false,true],[true,false]]);
Graph( Category := SimpleGraphs, Order := 2, Size := 1, Adjacencies :=
[ [ 2 ], [ 1 ] ] )
gap> G:=BoxTimesProduct(G,G);
Graph( Category := SimpleGraphs, Order := 4, Size := 6, Adjacencies :=
[ [ 2, 3, 4 ], [ 1, 3, 4 ], [ 1, 2, 4 ], [ 1, 2, 3 ] ] )
\endexample



$$
\vcenter{\xymatrix{
   {\bullet} \\
   {\bullet} \ar@{-}[u]
}}
\boxtimes
\vcenter{\xymatrix{
   {\bullet} \\
   {\bullet} \ar@{-}[u]
}}
=
\vcenter{\xymatrix{
   {\bullet} \ar@{-}[r] \ar@{-}[d] \ar@{-}[rd] & {\bullet}
   \ar@{-}[d] \ar@{-}[dl] \\ 
   {\bullet} \ar@{-}[r] & {\bullet} 
}}
$$

In the previous examples $k^2$ (<i.e.> the complete graph or order
two) was chosen because it better pictures how the operators work.

\>DisjointUnion( <G>, <H> ) O

Returns the DisjointUnion of two graphs <G> and <H>, <G> $\dot{\cup}$ <H>.

A disjoint union of graphs is obtained by combining both graphs in a
disjoint graph.

\beginexample
gap> G:=RandomGraph(4);
Graph( Category := Graphs, Order := 4, Size := 8, Adjacencies :=
[ [ 3, 4 ], [ 4 ], [ 1, 2, 3, 4 ], [ 2 ] ] )
gap> H:=RandomGraph(4);
Graph( Category := Graphs, Order := 4, Size := 8, Adjacencies :=
[ [ 3, 4 ], [ 4 ], [ 1, 2, 3, 4 ], [ 2 ] ] )
gap> DisjointUnion(G,H);
Graph( Category := Graphs, Order := 4, Size := 8, Adjacencies :=
[ [ 3, 4 ], [ 4 ], [ 1, 2, 3, 4 ], [ 2 ] ] )
\endexample



$$
\vcenter{\xymatrix{
   {\bullet} \ar@{-}[d] & {\bullet} \ar@{-}[l] \\
   {\bullet} \ar@{-}[ur]
}}
\dot{\cup}
\vcenter{\xymatrix{
   {\bullet} \\
   {\bullet} \ar@{-}[u]
}}
=
\vcenter{
\hbox{
\xymatrix{
   {\bullet} \ar@{-}[d] & {\bullet} \ar@{-}[l] \\
   {\bullet} \ar@{-}[ur]
}
\enskip
\xymatrix{
   {\bullet} \\
   {\bullet} \ar@{-}[u]
}}}
$$

\>Join( <G>, <H> ) O

Returns the result of joining graph <G> and <H>, <G> + <H>.

Joining graphs is computed as follows:

First, we obtain the disjoint union of graphs <G> and <H>. Second,
for each vertex $g \in G$ we add an edge to each vertex $h \in H.$
Finally, for each vertex $g \in G$ we add an edge to each vertex $h
\in H.$ 

\beginexample
gap> G:=RandomGraph(4);
Graph( Category := Graphs, Order := 4, Size := 8, Adjacencies :=
[ [ 3, 4 ], [ 4 ], [ 1, 2, 3, 4 ], [ 2 ] ] )
gap> H:=RandomGraph(4);
Graph( Category := Graphs, Order := 4, Size := 8, Adjacencies :=
[ [ 3, 4 ], [ 4 ], [ 1, 2, 3, 4 ], [ 2 ] ] )
gap> Join(G,H);
Graph( Category := Graphs, Order := 4, Size := 8, Adjacencies :=
[ [ 3, 4 ], [ 4 ], [ 1, 2, 3, 4 ], [ 2 ] ] )
\endexample



$$
\vcenter{\xymatrix{
   {\bullet} \ar@{-}[d] & {\bullet} \ar@{-}[l] \\
   {\bullet} \ar@{-}[ur]
}} \phantom{cm}
+ \phantom{cm}
\vcenter{\xymatrix{
   {\bullet} \\
   {\bullet} \ar@{-}[u]
}}
=
\vcenter{\xymatrix{
   {\bullet} \ar@{-}[d] & {\bullet} \ar@{-}[l] \ar@{-}[r] & 
              {\bullet} \ar@{-}[lld] \ar@(ul,ur)@{-}[ll] \\
   {\bullet} \ar@{-}[ur] \ar@{-}[rr] & & 
              {\bullet} \ar@{-}[u] \ar@{-}[llu] \ar@{-}[lu]
}}
$$

\>GraphSum( <G>, <L> ) O

Returns the GraphSum of <G> and a list of graphs <L>.

The GraphSum is computed as follows:

Given <G> and a list of graphs $L = L_1, \dots L_n$,
if <G> is the trivial graph the result is $L_1.$ Otherwise we take
$g_1,g_2 \in G.$ If $g_1 \sim g_2$ we compute $L_1 + L_2$ and the
DisjointUnion in other case. We repeat this process for every
$g_i, g_{i+1}$ until $i+1 = n.$

If a graph is not given in a particular element of the list the
trivial graph will be used, <e.g.> $[H, , J, ] \equiv [H, T, J,
T]$ where <T> is the trivial graph. 

\beginexample
gap> G:=RandomGraph(4);
Graph( Category := Graphs, Order := 4, Size := 8, Adjacencies :=
[ [ 3, 4 ], [ 4 ], [ 1, 2, 3, 4 ], [ 2 ] ] )
gap> H:=RandomGraph(4);
Graph( Category := Graphs, Order := 4, Size := 8, Adjacencies :=
[ [ 3, 4 ], [ 4 ], [ 1, 2, 3, 4 ], [ 2 ] ] )
gap> GraphSum(G,H);
Graph( Category := Graphs, Order := 4, Size := 8, Adjacencies :=
[ [ 3, 4 ], [ 4 ], [ 1, 2, 3, 4 ], [ 2 ] ] )
\endexample


\>Composition( <G>, <H> ) O

Returns the composition of two graphs <G> and <H>, $G[H].$

A composition of graphs is obtained by calculating the GraphSum
of <G> with <H>, $$G[H] = \hbox{GraphSum}(G, [H, \dots, H]).$$

\beginexample
gap> G:=RandomGraph(4);
Graph( Category := Graphs, Order := 4, Size := 8, Adjacencies :=
[ [ 3, 4 ], [ 4 ], [ 1, 2, 3, 4 ], [ 2 ] ] )
gap> H:=RandomGraph(4);
Graph( Category := Graphs, Order := 4, Size := 8, Adjacencies :=
[ [ 3, 4 ], [ 4 ], [ 1, 2, 3, 4 ], [ 2 ] ] )
gap> Composition(G,H);
Graph( Category := Graphs, Order := 4, Size := 8, Adjacencies :=
[ [ 3, 4 ], [ 4 ], [ 1, 2, 3, 4 ], [ 2 ] ] )
\endexample








