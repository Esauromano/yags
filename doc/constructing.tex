% This file was created automatically from constructing.msk.
% DO NOT EDIT!
\Chapter{Constructing graphs}

\Section{Primitives}

The following functions create new graphs from a variety of sources.

\>Graph( <R> ) O

Returns a new graph created from the record <R>. The record must provide the field <Category> 
and either the field <Adjacencies> or the field <AdjMatrix>

\beginexample
gap> Graph(rec(Category:=SimpleGraphs,Adjacencies:=[[2],[1]]));
Graph( Category := SimpleGraphs, Order := 2, Size := 1, Adjacencies := [ [ 2 ], [ 1 ] ] )
gap> Graph(rec(Category:=SimpleGraphs,AdjMatrix:=[[false, true],[true, false]]));
Graph( Category := SimpleGraphs, Order := 2, Size := 1, Adjacencies := [ [ 2 ], [ 1 ] ] )
\endexample

Its main purpose is to import graphs from files, which could have been 
previously exported using `PrintTo'.

\beginexample
gap> g:=CycleGraph(4);
Graph( Category := SimpleGraphs, Order := 4, Size := 4, Adjacencies := 
[ [ 2, 4 ], [ 1, 3 ], [ 2, 4 ], [ 1, 3 ] ] )
gap> PrintTo("aux.g","h1:=",g,";");
gap> Read("aux.g");
gap> h1;
Graph( Category := SimpleGraphs, Order := 4, Size := 4, Adjacencies := 
[ [ 2, 4 ], [ 1, 3 ], [ 2, 4 ], [ 1, 3 ] ] )
\endexample


\>GraphByAdjMatrix( <M> ) F

Returns a new graph created from an adjacency matrix <M>. The matrix <M> must
be a square boolean matrix.

\beginexample
gap> m:=[ [ false, true, false ], [ true, false, true ], [ false, true, false ] ];;
gap> g:=GraphByAdjMatrix(m);
Graph( Category := SimpleGraphs, Order := 3, Size := 2, Adjacencies := 
[ [ 2 ], [ 1, 3 ], [ 2 ] ] )
gap> AdjMatrix(g);
[ [ false, true, false ], [ true, false, true ], [ false, true, false ] ]
\endexample

Note, however, that the graph is forced to comply with the `TargetGraphCategory'.

\beginexample
gap> m:=[ [ true, true], [ false, false ] ];;
gap> g:=GraphByAdjMatrix(m);                
Graph( Category := SimpleGraphs, Order := 2, Size := 1, Adjacencies := [ [ 2 ], [ 1 ] ] )
gap> AdjMatrix(g);                          
[ [ false, true ], [ true, false ] ]
\endexample


\>GraphByAdjacencies( <A> ) F

Returns a new graph having <A> as its list of adjacencies. The order of the created graph is 
`Length(A)', and the set of neighbors of vertex $x$ is $A[x]$. 

\beginexample
gap> GraphByAdjacencies([[2],[1,3],[2]]);      
Graph( Category := SimpleGraphs, Order := 3, Size := 2, Adjacencies := 
[ [ 2 ], [ 1, 3 ], [ 2 ] ] )
\endexample

Note, however, that the graph is forced to comply with the `TargetGraphCategory'.

\beginexample
gap> GraphByAdjacencies([[1,2,3],[],[]]);
Graph( Category := SimpleGraphs, Order := 3, Size := 2, Adjacencies := 
[ [ 2, 3 ], [ 1 ], [ 1 ] ] )
\endexample


\>GraphByCompleteCover( <C> ) F

Returns the minimal graph where the elements of <C> are 
(the vertex sets of) complete subgraphs.

\beginexample
gap> GraphByCompleteCover([[1,2,3,4],[4,6,7]]); 
Graph( Category := SimpleGraphs, Order := 7, Size := 9, Adjacencies := 
[ [ 2, 3, 4 ], [ 1, 3, 4 ], [ 1, 2, 4 ], [ 1, 2, 3, 6, 7 ], [  ], [ 4, 7 ], 
  [ 4, 6 ] ] )
\endexample


\>GraphByRelation( <V>, <R> ) F
\>GraphByRelation( <N>, <R> ) F

Returns a new graph created from a set of vertices $V$ and a binary relation $R$, 
where $x\sim y$ iff $R(x,y)=true$. In the second form, $N$ is an integer
and $V$ is assumed to be $\{1, 2, \ldots, N\}$.

\beginexample
gap> R:=function(x,y) return Intersection(x,y)<>[]; end;;          
gap> GraphByRelation([[1,2,3],[3,4,5],[5,6,7]],R);               
Graph( Category := SimpleGraphs, Order := 3, Size := 2, Adjacencies := 
[ [ 2 ], [ 1, 3 ], [ 2 ] ] )
gap> GraphByRelation(8,function(x,y) return AbsInt(x-y)<=2; end); 
Graph( Category := SimpleGraphs, Order := 8, Size := 13, Adjacencies := 
[ [ 2, 3 ], [ 1, 3, 4 ], [ 1, 2, 4, 5 ], [ 2, 3, 5, 6 ], [ 3, 4, 6, 7 ], 
  [ 4, 5, 7, 8 ], [ 5, 6, 8 ], [ 6, 7 ] ] )
\endexample


\>GraphByWalks( <walk1>, <walk2>, ... ) F

Returns the minimal graph such that <walk1>, <walk2>, etc are walks.

\beginexample
gap> GraphByWalks([1,2,3,4,1],[1,5,6]);
Graph( Category := SimpleGraphs, Order := 6, Size := 6, Adjacencies := 
[ [ 2, 4, 5 ], [ 1, 3 ], [ 2, 4 ], [ 1, 3 ], [ 1, 6 ], [ 5 ] ] )
\endexample

Walks can be <nested>, which greatly improves the versatility of this function.
 
\beginexample
gap> GraphByWalks([1,[2,3,4],5],[5,6]);
Graph( Category := SimpleGraphs, Order := 6, Size := 9, Adjacencies := 
[ [ 2, 3, 4 ], [ 1, 3, 5 ], [ 1, 2, 4, 5 ], [ 1, 3, 5 ], [ 2, 3, 4, 6 ], [ 5 ] ] )
\endexample

The vertices in the constructed graph range from 1 to the maximum of the numbers
appearing in <walk1>, <walk2>, ... etc.

\beginexample
gap> GraphByWalks([4,2],[3,6]);
Graph( Category := SimpleGraphs, Order := 6, Size := 2, Adjacencies := 
[ [  ], [ 4 ], [ 6 ], [ 2 ], [  ], [ 3 ] ] )
\endexample


\>IntersectionGraph( <L> ) F

Returns the intersection graph of the family of sets <L>. This graph has a 
vertex for every set in <L>, and two such vertices are adjacent iff the 
corresponding sets have non-empty intersection.

\beginexample
gap> IntersectionGraph([[1,2,3],[3,4,5],[5,6,7]]);
Graph( Category := SimpleGraphs, Order := 3, Size := 2, Adjacencies := 
[ [ 2 ], [ 1, 3 ], [ 2 ] ] )
\endexample



The following functions create graphs from existing graphs

\>CopyGraph( <G> ) O

Returns a fresh copy of graph <G>. Only the order and adjacency information is copied, 
all other known attributes of <G> are not. Mainly used to transform a graph from one category 
to another. The new graph will be forced to comply with the `TargetGraphCategory'.

\beginexample
gap> g:=CompleteGraph(4);                         
Graph( Category := SimpleGraphs, Order := 4, Size := 6, Adjacencies := 
[ [ 2, 3, 4 ], [ 1, 3, 4 ], [ 1, 2, 4 ], [ 1, 2, 3 ] ] )
gap> g1:=CopyGraph(g:GraphCategory:=OrientedGraphs);
Graph( Category := OrientedGraphs, Order := 4, Size := 6, Adjacencies := 
[ [ 2, 3, 4 ], [ 3, 4 ], [ 4 ], [  ] ] )
gap> CopyGraph(g1:GraphCategory:=SimpleGraphs);     
Graph( Category := SimpleGraphs, Order := 4, Size := 6, Adjacencies := 
[ [ 2, 3, 4 ], [ 1, 3, 4 ], [ 1, 2, 4 ], [ 1, 2, 3 ] ] )
\endexample


\>InducedSubgraph( <G>, <V> ) O

Returns the subgraph of graph <G> induced by the vertex set <V>.

\beginexample
gap> g:=CycleGraph(6);          
Graph( Category := SimpleGraphs, Order := 6, Size := 6, Adjacencies := 
[ [ 2, 6 ], [ 1, 3 ], [ 2, 4 ], [ 3, 5 ], [ 4, 6 ], [ 1, 5 ] ] )
gap> InducedSubgraph(g,[3,4,6]);  
Graph( Category := SimpleGraphs, Order := 3, Size := 1, Adjacencies := 
[ [ 2 ], [ 1 ], [  ] ] )
\endexample

The order of the elements in <V> does matter.

\beginexample
gap> InducedSubgraph(g,[6,3,4]);  
Graph( Category := SimpleGraphs, Order := 3, Size := 1, Adjacencies := 
[ [  ], [ 3 ], [ 2 ] ] )
\endexample


\>RemoveVertices( <G>, <V> ) O

Returns a new graph created from graph <G> by removing the vertices in list <V>.

\beginexample
gap> g:=PathGraph(5);
Graph( Category := SimpleGraphs, Order := 5, Size := 4, Adjacencies := 
[ [ 2 ], [ 1, 3 ], [ 2, 4 ], [ 3, 5 ], [ 4 ] ] )
gap> RemoveVertices(g,[3]);
Graph( Category := SimpleGraphs, Order := 4, Size := 2, Adjacencies := 
[ [ 2 ], [ 1 ], [ 4 ], [ 3 ] ] )
gap> RemoveVertices(g,[1,3]);
Graph( Category := SimpleGraphs, Order := 3, Size := 1, Adjacencies := 
[ [  ], [ 3 ], [ 2 ] ] )
\endexample


\>AddEdges( <G>, <E> ) O

Returns a new graph created from graph <G> by adding the edges in list <E>.

\beginexample
gap> g:=CycleGraph(4);   
Graph( Category := SimpleGraphs, Order := 4, Size := 4, Adjacencies := 
[ [ 2, 4 ], [ 1, 3 ], [ 2, 4 ], [ 1, 3 ] ] )
gap> AddEdges(g,[[1,3]]);
Graph( Category := SimpleGraphs, Order := 4, Size := 5, Adjacencies := 
[ [ 2, 3, 4 ], [ 1, 3 ], [ 1, 2, 4 ], [ 1, 3 ] ] )
gap> AddEdges(g,[[1,3],[2,4]]);
Graph( Category := SimpleGraphs, Order := 4, Size := 6, Adjacencies := 
[ [ 2, 3, 4 ], [ 1, 3, 4 ], [ 1, 2, 4 ], [ 1, 2, 3 ] ] )
\endexample


\>RemoveEdges( <G>, <E> ) O

Returns a new graph created from graph <G> by removing the edges in list <E>.

\beginexample
gap> g:=CompleteGraph(4);
Graph( Category := SimpleGraphs, Order := 4, Size := 6, Adjacencies := 
[ [ 2, 3, 4 ], [ 1, 3, 4 ], [ 1, 2, 4 ], [ 1, 2, 3 ] ] )
gap> RemoveEdges(g,[[1,2]]);
Graph( Category := SimpleGraphs, Order := 4, Size := 5, Adjacencies := 
[ [ 3, 4 ], [ 3, 4 ], [ 1, 2, 4 ], [ 1, 2, 3 ] ] )
gap> RemoveEdges(g,[[1,2],[3,4]]);
Graph( Category := SimpleGraphs, Order := 4, Size := 4, Adjacencies := 
[ [ 3, 4 ], [ 3, 4 ], [ 1, 2 ], [ 1, 2 ] ] )
\endexample



\>CliqueGraph( <G> ) A
\>CliqueGraph( <G>, <m> ) O

Returns the intersection graph of all the (maximal) cliques of <G>.

The additional parameter <m> aborts the computation when <m> cliques are found, 
even if they are all the cliques of <G>. If the bound <m> is reached, <fail> is returned.

\beginexample
gap> CliqueGraph(Octahedron);   
Graph( Category := SimpleGraphs, Order := 8, Size := 24, Adjacencies := 
[ [ 2, 3, 4, 5, 6, 7 ], [ 1, 3, 4, 5, 6, 8 ], [ 1, 2, 4, 5, 7, 8 ], 
  [ 1, 2, 3, 6, 7, 8 ], [ 1, 2, 3, 6, 7, 8 ], [ 1, 2, 4, 5, 7, 8 ], 
  [ 1, 3, 4, 5, 6, 8 ], [ 2, 3, 4, 5, 6, 7 ] ] )
gap> CliqueGraph(Octahedron,9); 
Graph( Category := SimpleGraphs, Order := 8, Size := 24, Adjacencies := 
[ [ 2, 3, 4, 5, 6, 7 ], [ 1, 3, 4, 5, 6, 8 ], [ 1, 2, 4, 5, 7, 8 ], 
  [ 1, 2, 3, 6, 7, 8 ], [ 1, 2, 3, 6, 7, 8 ], [ 1, 2, 4, 5, 7, 8 ], 
  [ 1, 3, 4, 5, 6, 8 ], [ 2, 3, 4, 5, 6, 7 ] ] )
gap> CliqueGraph(Octahedron,8);
fail
\endexample



\Section{Families}

The following functions return well known graphs. Most of them can be
found in Brandstadt, Le and Spinrad.

\>DiscreteGraph( <n> ) F

Returns the discrete graph of order <n>. A discrete graph is a graph
without edges.

\beginexample
gap> DiscreteGraph(4);
Graph( Category := SimpleGraphs, Order := 4, Size := 0, Adjacencies :=
[ [  ], [  ], [  ], [  ] ] )
\endexample



4-Discrete Graph
\enskip
$\vcenter{\xymatrix{
   {\bullet} & {\bullet} \\
   {\bullet} & {\bullet} 
}}$

\>CompleteGraph( <n> ) F

Returns the complete graph of order <n>. A complete graph is a graph
where all vertices are connected to each other.

\beginexample
gap> CompleteGraph(4);
Graph( Category := SimpleGraphs, Order := 4, Size := 6, Adjacencies :=
[ [ 2, 3, 4 ], [ 1, 3, 4 ], [ 1, 2, 4 ], [ 1, 2, 3 ] ] )
\endexample



4-Complete Graph
\enskip
$\vcenter{\xymatrix{
   {\bullet} \ar@{-}[r] \ar@{-}[d] & {\bullet} \ar@{-}[d] \ar@{-}[dl]\\
   {\bullet} \ar@{-}[r] & {\bullet} \ar@{-}[ul]
}}$

\>PathGraph( <n> ) F

Returns the path graph on <n> vertices.

\beginexample
gap> PathGraph(4);
Graph( Category := SimpleGraphs, Order := 4, Size := 3, Adjacencies :=
[ [ 2 ], [ 1, 3 ], [ 2, 4 ], [ 3 ] ] )
\endexample



4-Path Graph
\enskip
$\vcenter{\xymatrix{
   {\bullet} \ar@{-}[r] & {\bullet} \ar@{-}[r] &
   {\bullet} \ar@{-}[r] & {\bullet} 
}}$

\>CycleGraph( <n> ) F

Returns the cyclic graph on <n> vertices. 

\beginexample
gap> CycleGraph(5);
Graph( Category := SimpleGraphs, Order := 5, Size := 5, Adjacencies :=
[ [ 2, 5 ], [ 1, 3 ], [ 2, 4 ], [ 3, 5 ], [ 1, 4 ] ] )
\endexample


5-Cycle Graph
\enskip
\xy /r10mm/:
\xypolygon5{@{*}}
\endxy

\>CubeGraph( <n> ) F

Returns the hypercube of dimension <n>. This is the box product (cartesian product) 
of $n$ copies of $K_2$ (an edge).

\beginexample
gap> CubeGraph(3);
Graph( Category := SimpleGraphs, Order := 8, Size := 12, Adjacencies :=
[ [ 2, 3, 5 ], [ 1, 4, 6 ], [ 1, 4, 7 ], [ 2, 3, 8 ], [ 1, 6, 7 ],
[ 2, 5, 8 ], [ 3, 5, 8 ], [ 4, 6, 7 ] ] )
\endexample

-map

3-Cube Graph
\enskip
\objectmargin={1pt}
\xygraph{
 \bullet-[r]\bullet-[d]\bullet-[l]\bullet-[u],
  -[ur(.5)]\bullet-[r]\bullet-[d]\bullet-[l]\bullet-[u],
  [r]-[ur(.5)],
  [r][d]-[ur(.5)],
  [r][d][l]-[ur(.5)],
}

\>OctahedralGraph( <n> ) F

Return the <n>-dimensional octahedron. This is the complement of <n> copies
of $K_2$ (an edge). It is also the <(2n-2)>-regular graph on $2n$ vertices.

\beginexample
gap> OctahedralGraph(3);
Graph( Category := SimpleGraphs, Order := 6, Size := 12, Adjacencies :=
[ [ 3, 4, 5, 6 ], [ 3, 4, 5, 6 ], [ 1, 2, 5, 6 ], [ 1, 2, 5, 6 ],
[ 1, 2, 3, 4 ], [ 1, 2, 3, 4 ] ] )
\endexample


3-Octahedral Graph
\enskip
\objectmargin={1pt}
\xygraph{
 \bullet
   ([u][r(.5)][ur(.25)]\bullet-?,
    [d][r(.5)][ur(.25)]\bullet-?,
    -[r]\bullet,
    -[ur(.5)]\bullet,
   )
 [r]
   ([u][l(.5)][ur(.25)]-?,
    [d][l(.5)][ur(.25)]-?,
   )
 -[ur(.5)]\bullet
   ([u][l(.5)][dl(.25)]-?,
    [d][l(.5)][dl(.25)]-?,
   )
 -[l]
   ([u][r(.5)][dl(.25)]-?,
    [d][r(.5)][dl(.25)]-?,
   )
}

\>JohnsonGraph( <n>, <r> ) F

Returns the Johnson graph $J(n,r).$ A Johnson Graph is a 
graph constructed as follows. Each vertex represents a subset of
the set $\{1,\dots,n\}$ with cardinality $r.$ $$V(J(n,r)) = \{ X \subset
\{1,\dots,n\} | |X| = r \}$$ and there is an edge between two
vertices if and only if the cardinality of the intersection of the
sets they represent is $r-1$ $$X \sim X' \hbox{ iff } |X \cup X'| = r-1.$$

\beginexample
gap> JohnsonGraph(4,2);
Graph( Category := SimpleGraphs, Order := 6, Size := 12, Adjacencies :=
[ [ 2, 3, 4, 5 ], [ 1, 3, 4, 6 ], [ 1, 2, 5, 6 ], [ 1, 2, 5, 6 ],
 [ 1, 3, 4, 6 ], [ 2, 3, 4, 5 ] ] )
\endexample


\vskip .5cm
4,2-Johnson Graph
\enskip
$\vcenter{\xymatrix{
   {\bullet} \ar@{-}[r] \ar@{-}[d] \ar@{-}[dr]
     & {\bullet} \ar@{-}[r] \ar@{-}[dr] \ar@{-}[dl]
       & {\bullet} \ar@{-}[d] \ar@{-}[dl] \ar@(ul,ur)@{-}[ll] \\
   {\bullet} \ar@{-}[r]
     & {\bullet} \ar@{-}[r]
       & {\bullet} \ar@(dl,dr)@{-}[ll]
}}$

\phantom{J}

\>CompleteBipartiteGraph( <n>, <m> ) F

Returns the complete bipartite whose parts have order <n> and <m> respectively. 
This is the joint (Zykov sum) of two discrete graphs of order <n> and <m>. 

\beginexample
gap> CompleteBipartiteGraph(2,3);
Graph( Category := SimpleGraphs, Order := 5, Size := 6, Adjacencies :=
[ [ 3, 4, 5 ], [ 3, 4, 5 ], [ 1, 2 ], [ 1, 2 ], [ 1, 2 ] ] )
\endexample


2,3-Complete Bipartite Graph
\enskip
$\vcenter{
\xygraph{
 \bullet[r][u(.5)]\bullet[d]\bullet[d]\bullet
   [l][u(.5)]\bullet,
 ([r][u(.5)]-?,
  [r][u(.5)][d]-?,
  [r][u(.5)][d][d]-?,
 ),
 [d]
  ([r][u(1.5)]-?,
   [r][u(.5)]-?,
   [r][d(.5)]-?,
  )
}
}$

\>CompleteMultipartiteGraph( <n1>, <n2> [, <n3> ...] ) F

Returns the complete multipartite graph where the orders of the parts are 
<n1>, <n2>, ... It is also the Zykov sum of discrete graphs of order <n1>, <n2>, ...

\beginexample
gap> CompleteMultipartiteGraph(2,2,2);
Graph( Category := SimpleGraphs, Order := 6, Size := 12, Adjacencies :=
[ [ 3, 4, 5, 6 ], [ 3, 4, 5, 6 ], [ 1, 2, 5, 6 ], [ 1, 2, 5, 6 ],
 [ 1, 2, 3, 4 ], [ 1, 2, 3, 4 ] ] )
\endexample


2,2,2-Complete Multipartite Graph
\enskip
$\vcenter{
\xygraph{
 \bullet[d(.5)]\bullet[rd(.5)]\bullet[r(.5)]\bullet[ru(.5)]
   \bullet[u(.5)]\bullet,
 ([d(.5)][rd(.5)]-?,
  [d(.5)][rd(.5)][r(.5)]-?,
  [d(.5)][rd(.5)][r(.5)][ru(.5)]-?,
  [d(.5)][rd(.5)][r(.5)][ru(.5)][u(.5)]-?,
 ),
 [d(.5)]
  ([rd(.5)]-?,
   [rd(.5)][r(.5)]-?,
   [rd(.5)][r(.5)][ru(.5)]-?,
   [rd(.5)][r(.5)][ru(.5)][u(.5)]-?,
  ),
 [d(.5)][rd(.5)]
  ([r(.5)][ru(.5)]-?,
   [r(.5)][ru(.5)][u(.5)]-?,
  ),
 [d(.5)][rd(.5)][r(.5)]
  ([ru(.5)]-?,
   [ru(.5)][u(.5)]-?,
  )
}
}$

\>RandomGraph( <n>, <p> ) F
\>RandomGraph( <n> ) F

Returns a random graph of order <n> taking the rational $p\in [0,1]$ as the edge probability. 

\beginexample
gap> RandomGraph(5,1/3);
Graph( Category := SimpleGraphs, Order := 5, Size := 2, Adjacencies := 
[ [ 5 ], [ 5 ], [  ], [  ], [ 1, 2 ] ] )
gap> RandomGraph(5,2/3);
Graph( Category := SimpleGraphs, Order := 5, Size := 6, Adjacencies := 
[ [ 4, 5 ], [ 3, 4, 5 ], [ 2, 4 ], [ 1, 2, 3 ], [ 1, 2 ] ] )
gap> RandomGraph(5,1/2);
Graph( Category := SimpleGraphs, Order := 5, Size := 4, Adjacencies := 
[ [ 2, 5 ], [ 1, 3, 5 ], [ 2 ], [  ], [ 1, 2 ] ] )
\endexample

If <p> is ommited, the edge probability is taken to be 1/2.

\beginexample
gap> RandomGraph(5);    
Graph( Category := SimpleGraphs, Order := 5, Size := 5, Adjacencies := 
[ [ 2, 3 ], [ 1 ], [ 1, 4, 5 ], [ 3, 5 ], [ 3, 4 ] ] )
gap> RandomGraph(5);
Graph( Category := SimpleGraphs, Order := 5, Size := 3, Adjacencies := 
[ [ 2, 5 ], [ 1, 4 ], [  ], [ 2 ], [ 1 ] ] )
\endexample


5-Random Graph

\>WheelGraph( <N> ) O
\>WheelGraph( <N>, <Radius> ) O

In its first form `WheelGraph' returns the wheel graph on <N+1> vertices. This is the 
cone of a cycle: a central vertex adjacent to all the vertices of an <N>-cycle

\beginexample
WheelGraph(5);
gap> Graph( Category := SimpleGraphs, Order := 6, Size := 10, Adjacencies :=
[ [ 2, 3, 4, 5, 6 ], [ 1, 3, 6 ], [ 1, 2, 4 ], [ 1, 3, 5 ], [ 1, 4, 6 ],
 [ 1, 2, 5 ] ] )
\endexample

In its second form, `WheelGraph' returns returns the wheel graph, but adding 
<Radius-1> layers, each layer is a new <N>-cycle joined to the previous layer 
by a zigzagging <2N>-cycle. This graph is a triangulation of the disk.

\beginexample
gap> WheelGraph(5,2);
Graph( Category := SimpleGraphs, Order := 11, Size := 25, Adjacencies := 
[ [ 2, 3, 4, 5, 6 ], [ 1, 3, 6, 7, 8 ], [ 1, 2, 4, 8, 9 ], [ 1, 3, 5, 9, 10 ],
  [ 1, 4, 6, 10, 11 ], [ 1, 2, 5, 7, 11 ], [ 2, 6, 8, 11 ], [ 2, 3, 7, 9 ], 
  [ 3, 4, 8, 10 ], [ 4, 5, 9, 11 ], [ 5, 6, 7, 10 ] ] )
gap> WheelGraph(5,3);
Graph( Category := SimpleGraphs, Order := 16, Size := 40, Adjacencies := 
[ [ 2, 3, 4, 5, 6 ], [ 1, 3, 6, 7, 8 ], [ 1, 2, 4, 8, 9 ], [ 1, 3, 5, 9, 10 ],
  [ 1, 4, 6, 10, 11 ], [ 1, 2, 5, 7, 11 ], [ 2, 6, 8, 11, 12, 13 ], 
  [ 2, 3, 7, 9, 13, 14 ], [ 3, 4, 8, 10, 14, 15 ], [ 4, 5, 9, 11, 15, 16 ], 
  [ 5, 6, 7, 10, 12, 16 ], [ 7, 11, 13, 16 ], [ 7, 8, 12, 14 ], 
  [ 8, 9, 13, 15 ], [ 9, 10, 14, 16 ], [ 10, 11, 12, 15 ] ] )
\endexample


\undogap
Wheel Graph of Order 5
\enskip
\xy /r10mm/:,
\drop{\bullet}\xypolygon5{~<{-}\bullet}
\endxy
\dogap

\>FanGraph( <N> ) F

Returns the <N>-Fan: The join of a vertex and a <(N+1)>-path.

\beginexample
gap> FanGraph(4);
Graph( Category := SimpleGraphs, Order := 6, Size := 9, Adjacencies :=
[ [ 2, 3, 4, 5, 6 ], [ 1, 3 ], [ 1, 2, 4 ], [ 1, 3, 5 ], [ 1, 4, 6 ],
[ 1, 5 ] ] )
\endexample


4-Fan Graph
\enskip
$\vcenter{
\objectmargin={1pt}
\xygraph{
 \bullet (-[u]\bullet-[r(.6)]-[dr(.4)]-?
   ,[u][r(.6)]\bullet-?
   ,-[r]\bullet-[u(.6)]\bullet-?
}
}$

\>SunGraph( <N> ) F

Returns the <N>-Sun: A complete graph on <N> vertices, $K_N$, with a corona 
made with a zigzagging <2N>-cycle glued to a <N>-cyle of the $K_N$.

\beginexample
gap> SunGraph(3);
Graph( Category := SimpleGraphs, Order := 6, Size := 9, Adjacencies := 
[ [ 2, 6 ], [ 1, 3, 4, 6 ], [ 2, 4 ], [ 2, 3, 5, 6 ], [ 4, 6 ], 
  [ 1, 2, 4, 5 ] ] )
gap> SunGraph(4);
Graph( Category := SimpleGraphs, Order := 8, Size := 14, Adjacencies := 
[ [ 2, 8 ], [ 1, 3, 4, 6, 8 ], [ 2, 4 ], [ 2, 3, 5, 6, 8 ], [ 4, 6 ], 
  [ 2, 4, 5, 7, 8 ], [ 6, 8 ], [ 1, 2, 4, 6, 7 ] ] )
\endexample


4-Sun Graph
\enskip
$\vcenter{
\objectmargin={1pt}
\xygraph{
 \bullet-[dr]\bullet-[dl]\bullet-[ul]\bullet-[ur],
 [dr(.5)]\bullet-[d]\bullet-[l]\bullet-[u]\bullet-[r],
 [dr(.5)]-[dl][r]-[ul]
}
}$

\>SpikyGraph( <N> ) F

The spiky graph is constructed as follows: Take complete graph on <N> vertices, $K_N$, 
and then, for each the <N> subsets of $Vertices(K_n)$ of order <N-1>, add an additional vertex which
is adjacent precisely to this subset of $Vertices(K_n)$.

\beginexample
gap> SpikyGraph(3);
Graph( Category := SimpleGraphs, Order := 6, Size := 9, Adjacencies :=
[ [ 2, 3, 4, 5 ], [ 1, 3, 4, 6 ], [ 1, 2, 5, 6 ], [ 1, 2 ], [ 1, 3 ],
  [ 2, 3 ] ] )
\endexample


3-Spiky Graph
\enskip
$\vcenter{
\objectmargin={1pt}
\xygraph{
 \bullet,-[r(.5)]\bullet-[dr(.5)]\bullet-[u(.5)]\bullet-[l(.5)]\bullet
  -[dl(.5)]\bullet-[r],
  -[d(.5)]-[dr(.5)]\bullet-[ur(.5)]
}
}$
\>`TrivialGraph' V

The one vertex graph.

\beginexample
gap> TrivialGraph;
Graph( Category := SimpleGraphs, Order := 1, Size := 0, Adjacencies :=
[ [  ] ] )
\endexample


Trivial Graph
\xygraph{
 \bullet
}


\>`DiamondGraph' V

The graph on 4 vertices and 5 edges.

\beginexample
gap> DiamondGraph;
Graph( Category := SimpleGraphs, Order := 4, Size := 5, Adjacencies :=
[ [ 2, 3, 4 ], [ 1, 3 ], [ 1, 2, 4 ], [ 1, 3 ] ] )
\endexample


Diamond Graph
\enskip
$\vcenter{
\objectmargin={1pt}
\xygraph{
 \bullet,-[dl(.5)]\bullet-[dr(.5)]\bullet-[ur(.5)]\bullet-[l]\bullet,
  -[dr(.5)]\bullet
}
}$

\>`ClawGraph' V

The graph on 4 vertices, 3 edges, and maximum degree 3. 

\beginexample
gap> ClawGraph;
Graph( Category := SimpleGraphs, Order := 4, Size := 3, Adjacencies :=
[ [ 2, 3, 4 ], [ 1 ], [ 1 ], [ 1 ] ] )
\endexample


Claw Graph
\objectmargin={1pt}
\xygraph{
 \bullet,-[u(.5)]\bullet,-[dl(.5)]\bullet,-[dr(.5)]\bullet
}

\>`PawGraph' V

The graph on 4 vertices, 4 edges and maximum degree 3: A triangle with a pendant vertex.

\beginexample
gap> PawGraph;
Graph( Category := SimpleGraphs, Order := 4, Size := 4, Adjacencies :=
[ [ 2 ], [ 1, 3, 4 ], [ 2, 4 ], [ 2, 3 ] ] )
\endexample


Paw Graph
\enskip
$\vcenter{
\objectmargin={1pt}
\xygraph{
 \bullet,-[d(.5)]\bullet-[dl(.5)]\bullet-[r]\bullet-[ul(.5)]
}
}$

\>`HouseGraph' V

A 4-Cycle and a triangle glued by an edge.

\beginexample
gap> HouseGraph;
Graph( Category := SimpleGraphs, Order := 5, Size := 6, Adjacencies := 
[ [ 2, 4, 5 ], [ 1, 3 ], [ 2, 4 ], [ 1, 3, 5 ], [ 1, 4 ] ] )
\endexample


House Graph
\enskip
$\vcenter{
\objectmargin={1pt}
\xygraph{
 \bullet,-[dl(.5)]\bullet-[d(.5)]\bullet-[r]\bullet-[u(.5)]\bullet-[l]\bullet,
  -[dr(.5)]\bullet
}
}$

\>`BullGraph' V

A triangle with two pendant vertices (horns).

\beginexample
gap> BullGraph;    
Graph( Category := SimpleGraphs, Order := 5, Size := 5, Adjacencies := 
[ [ 2 ], [ 1, 3, 4 ], [ 2, 4 ], [ 2, 3, 5 ], [ 4 ] ] )
\endexample


Bull Graph
\enskip
$\vcenter{
\objectmargin={1pt}
\xygraph{
 \bullet,-[ul(.5)]\bullet-[u(.5)]\bullet,
  -[ur(.5)]\bullet-[u(.5)]\bullet,
  [ur(.5)]-[l]\bullet
}
}$

\>`AntennaGraph' V

A `HouseGraph' with a pendant vertex (antenna) on the roof.

\beginexample
gap> AntennaGraph;
Graph( Category := SimpleGraphs, Order := 6, Size := 7, Adjacencies := 
[ [ 2, 4, 5 ], [ 1, 3 ], [ 2, 4 ], [ 1, 3, 5 ], [ 1, 4, 6 ], [ 5 ] ] )
\endexample


Antenna Graph
\enskip
$\vcenter{
\objectmargin={1pt}
\xygraph{
 \bullet,-[dl(.5)]\bullet-[d(.5)]\bullet-[r]\bullet-[u(.5)]\bullet-[l]\bullet,
  -[dr(.5)]\bullet,
  -[u(.5)]\bullet
}
}$

\>`KiteGraph' V

A diamond with a pending vertex and maximum degree 3.

\beginexample
gap> KiteGraph;
Graph( Category := SimpleGraphs, Order := 5, Size := 6, Adjacencies := 
[ [ 2 ], [ 1, 3, 4 ], [ 2, 4, 5 ], [ 2, 3, 5 ], [ 3, 4 ] ] )
\endexample


Kite Graph
\objectmargin={1pt}
\xygraph{
 \bullet,-[l(.5)]\bullet,
 -[dr(.5)]\bullet-[ur(.5)]\bullet-[ul(.5)]\bullet-[dl(.5)],
 [dr(.5)]-[u]
}

\>`Tetrahedron' V

The 1-skeleton of Plato's tetrahedron.

\beginexample
gap> Tetrahedron;
Graph( Category := SimpleGraphs, Order := 4, Size := 6, Adjacencies :=
[ [ 2, 3, 4 ], [ 1, 3, 4 ], [ 1, 2, 4 ], [ 1, 2, 3 ] ] )
\endexample


Tetrahedron
\xygraph{ %/r10mm/:
 \bullet (-[u(.5)]\bullet,-[dr(.5)]\bullet,-[dl(.5)]\bullet),
 [u(.5)] ([d(.5)][dr(.5)]-?, [d(.5)][dl(.5)]-?),
 [dr(.5)]-[l]
}


\>`Octahedron' V

The 1-skeleton of Plato's octahedron.

\beginexample
gap> Octahedron;
Graph( Category := SimpleGraphs, Order := 6, Size := 12, Adjacencies :=
[ [ 3, 4, 5, 6 ], [ 3, 4, 5, 6 ], [ 1, 2, 5, 6 ], [ 1, 2, 5, 6 ],
  [ 1, 2, 3, 4 ], [ 1, 2, 3, 4 ] ] )
\endexample


\undogap
Octahedron
\enskip
$\vcenter{
\xy /r2pc/:
{\xypolygon3"A"{~:{(0,.7)::}\bullet}},+(.7,1.1),
{\xypolygon3"B"{~:{(-.85,0):(-.150,.8)::}\bullet}},
{"A1"\PATH~={**@{-}}'"B2"'"A3"'"B1"'"A2"'"B3"'"A1"}
\endxy
}$
\dogap

\>`Cube' V

The 1-skeleton of Plato's cube.

\beginexample
gap> Cube;
Graph( Category := SimpleGraphs, Order := 8, Size := 12, Adjacencies :=
[ [ 2, 3, 5 ], [ 1, 4, 6 ], [ 1, 4, 7 ], [ 2, 3, 8 ], [ 1, 6, 7 ],
 [ 2, 5, 8 ], [ 3, 5, 8 ], [ 4, 6, 7 ] ] )
\endexample


Cube Graph
\enskip
$\vcenter{
\objectmargin={1pt}
\xygraph{
 \bullet-[r]\bullet-[d]\bullet-[l]\bullet-[u],
  -[ur(.5)]\bullet-[r]\bullet-[d]\bullet-[l]\bullet-[u],
  [r]-[ur(.5)],
  [r][d]-[ur(.5)],
  [r][d][l]-[ur(.5)],
}
}$

\>`Icosahedron' V

The 1-skeleton of Plato's icosahedron.

\beginexample
gap> Icosahedron;
Graph( Category := SimpleGraphs, Order := 12, Size := 30, Adjacencies :=
[ [ 2, 3, 4, 5, 6 ], [ 1, 3, 6, 9, 10 ], [ 1, 2, 4, 10, 11 ],
  [ 1, 3, 5, 7, 11 ], [ 1, 4, 6, 7, 8 ], [ 1, 2, 5, 8, 9 ],
  [ 4, 5, 8, 11, 12 ], [ 5, 6, 7, 9, 12 ], [ 2, 6, 8, 10, 12 ],
  [ 2, 3, 9, 11, 12 ], [ 3, 4, 7, 10, 12 ], [ 7, 8, 9, 10, 11 ] ] )
\endexample


\undogap
Icosahedron
\xy /r3pc/:
 ="S",
 +(0,1.5)*{\bullet}="A",
 "S",*{\bullet},
 {\xypolygon5"B"{~:{(1.5,0):(0,.33)::}~<>{;"A"**@{-}}\bullet}},
 {\xypolygon5"C"{~={55}~:{(.7,0):(0,.33)::}~<{-}\bullet}},
 {"B1"\PATH~={**@{-}}'"C1"},{"B1"\PATH~={**@{-}}'"C5"},
 {"B2"\PATH~={**@{-}}'"C1"},{"B2"\PATH~={**@{-}}'"C2"},
 {"B3"\PATH~={**@{-}}'"C2"},{"B3"\PATH~={**@{-}}'"C3"},
 {"B4"\PATH~={**@{-}}'"C3"},{"B4"\PATH~={**@{-}}'"C4"},
 {"B5"\PATH~={**@{-}}'"C4"},{"B5"\PATH~={**@{-}}'"C5"}
\endxy
\dogap

\>`Dodecahedron' V

The 1-skeleton of Plato's Dodecahedron.

\beginexample
gap> Dodecahedron;
Graph( Category := SimpleGraphs, Order := 20, Size := 30, Adjacencies := 
[ [ 2, 5, 6 ], [ 1, 3, 7 ], [ 2, 4, 8 ], [ 3, 5, 9 ], [ 1, 4, 10 ], 
  [ 1, 11, 15 ], [ 2, 11, 12 ], [ 3, 12, 13 ], [ 4, 13, 14 ], [ 5, 14, 15 ], 
  [ 6, 7, 16 ], [ 7, 8, 17 ], [ 8, 9, 18 ], [ 9, 10, 19 ], [ 6, 10, 20 ], 
  [ 11, 17, 20 ], [ 12, 16, 18 ], [ 13, 17, 19 ], [ 14, 18, 20 ], 
  [ 15, 16, 19 ] ] )
\endexample


\undogap
Dodecahedron
\enskip
\xy /l1.5pc/:,
  {\xypolygon5"A"{\bullet}},
  {\xypolygon5"B"{~:{(1.875,0):}~>{}\bullet}},
  {\xypolygon5"C"{~:{(-2.95,0):}~>{}\bullet}},
  {\xypolygon5"D"{~:{(-3.75,0):}\bullet}},
  {"A1"\PATH~={**@{-}}'"B1"'"C4"'"B2"},
  {"A2"\PATH~={**@{-}}'"B2"'"C5"'"B3"},
  {"A3"\PATH~={**@{-}}'"B3"'"C1"'"B4"},
  {"A4"\PATH~={**@{-}}'"B4"'"C2"'"B5"},
  {"A5"\PATH~={**@{-}}'"B5"'"C3"'"B1"},
  "C1";"D1"**@{-},"C2";"D2"**@{-},
  "C3";"D3"**@{-},"C4";"D4"**@{-},
  "C5";"D5"**@{-}
\endxy

%ar@{-}[r] \ar@{-}[dar@{-}[r] \ar@{-}[d]
%ar@{-}[r] \ar@{-}[d]

%\xymatrix{
%   {\bullet}  & {\bullet} \ar@{-}[d]\\
%   {\bullet} \ar@{-}[r] & {\bullet} 
%}  

\Section{Unary operations}

These are operations that can be performed over graphs.

\>LineGraph( <G> ) O

Returns the line graph <L(G)> of graph <G>. The line graph is the
intersection graph of the edges of <G>, <i.e.> the vertices of
$L(G)$ are the edges of <G> two of them being adjacent iff they
are incident. 

\beginexample 
gap> g:=Tetrahedron;
Graph( Category := SimpleGraphs, Order := 4, Size := 6, Adjacencies :=
[ [ 2, 3, 4 ], [ 1, 3, 4 ], [ 1, 2, 4 ], [ 1, 2, 3 ] ] )
gap> LineGraph(g);
Graph( Category := SimpleGraphs, Order := 6, Size := 12, Adjacencies :=
[ [ 2, 3, 4, 5 ], [ 1, 3, 4, 6 ], [ 1, 2, 5, 6 ], [ 1, 2, 5, 6 ],
  [ 1, 3, 4, 6 ], [ 2, 3, 4, 5 ] ] )
\endexample


LineGraph(\xygraph{ %/r10mm/:
 \bullet (-[u(.5)]\bullet,-[dr(.5)]\bullet,-[dl(.5)]\bullet),
 [u(.5)] ([d(.5)][dr(.5)]-?, [d(.5)][dl(.5)]-?),
 [dr(.5)]-[l]
}) = 
\xygraph{[u(.5)]
 \bullet-[r]\bullet-[dl(.5)]\bullet-[dl(.5)]\bullet
   -[ul(.5)]\bullet-[ul(.5)]\bullet-[r]-[dl(.5)]-[r]-[ul(.5)],
 [r(.5)][ur(.5)] ([ll]-@(ur,ul)?, [dl]-@(r,d)?),
 [l(.5)][ul(.5)] ([dr]-@(l,d)?)
}

\>ComplementGraph( <G> ) A

Computes the complement of graph <G>. The complement of a graph is
created as follows:
Create a graph <G'> with same vertices of <G>. For each <x>, <y>
$\in$ <G> if <x> $\nsim$ <y> in <G> then <x> $\sim$ <y> in <G'>

\beginexample 
gap> g:=ClawGraph;
Graph( Category := SimpleGraphs, Order := 4, Size := 3, Adjacencies :=
[ [ 2, 3, 4 ], [ 1 ], [ 1 ], [ 1 ] ] )
gap> ComplementGraph(g);
Graph( Category := SimpleGraphs, Order := 4, Size := 3, Adjacencies :=
[ [  ], [ 3, 4 ], [ 2, 4 ], [ 2, 3 ] ] )
\endexample


ComplementGraph$(\vcenter{\xymatrix{
   {\bullet} \ar@{-}[r] \ar@{-}[d] & {\bullet} \ar@{-}[d]\\
   {\bullet} \ar@{-}[r] & {\bullet} 
}}
) = 
\vcenter{\xymatrix{
   {\bullet} \ar@{-}[rd] & {\bullet} \ar@{-}[dl]\\
   {\bullet} & {\bullet} 
}}$

\>QuotientGraph( <G>, <P> ) O
\>QuotientGraph( <G>, <L1>, <L2> ) O

Returns the quotient graph of graph <G> given a vertex partition
<P>, by identifying any two vertices in the same part. 
The vertices of the quotient
graph are the parts in the partition <P> two of them
being adjacent iff any vertex in one part is adjacent to any vertex in the other part.
Singletons may be omited in P.

\beginexample 
gap> g:=PathGraph(8);; 
gap> QuotientGraph(g,[[1,5,8],[2],[3],[4],[6],[7]]);
Graph( Category := SimpleGraphs, Order := 6, Size := 7, Adjacencies := 
[ [ 2, 4, 5, 6 ], [ 1, 3 ], [ 2, 4 ], [ 1, 3 ], [ 1, 6 ], [ 1, 5 ] ] )
gap> QuotientGraph(g,[[1,5,8]]);  
Graph( Category := SimpleGraphs, Order := 6, Size := 7, Adjacencies := 
[ [ 2, 4, 5, 6 ], [ 1, 3 ], [ 2, 4 ], [ 1, 3 ], [ 1, 6 ], [ 1, 5 ] ] )
\endexample

In its second form, `QuotientGraph' identifies each vertex in list <L1>, 
with the corresponding vertex in list <L2>. <L1> and <L2> must have the same length,
but any or both of them may have repetitions.

\beginexample 
gap> g:=PathGraph(8);; 
gap> QuotientGraph(g,[[1,7],[4,8]]);
Graph( Category := SimpleGraphs, Order := 6, Size := 7, Adjacencies := 
[ [ 2, 4, 6 ], [ 1, 3 ], [ 2, 4 ], [ 1, 3, 5 ], [ 4, 6 ], [ 1, 5 ] ] )
gap> QuotientGraph(g,[1,4],[7,8]);  
Graph( Category := SimpleGraphs, Order := 6, Size := 7, Adjacencies := 
[ [ 2, 4, 6 ], [ 1, 3 ], [ 2, 4 ], [ 1, 3, 5 ], [ 4, 6 ], [ 1, 5 ] ] )
\endexample


\undogap
\objectmargin={1pt}
QuotientGraph$(\vcenter{\xygraph{
 \bullet([ur(.1)]*{\scriptstyle 1})
  -[dl(.5)]\bullet([l(.1)]*{\scriptstyle 2})
  -[d(.5)]\bullet([l(.1)]*{\scriptstyle 3})
  -[r]\bullet([r(.1)]*{\scriptstyle 4})
  -[u(.5)]\bullet([r(.1)]*{\scriptstyle 5})
   ([l]\bullet-?
    [ul(.5)]\bullet-?
   )
}}, [2,3],[4,5]) =
\vcenter{\xygraph{
 [u(.2)]\bullet([ur(.1)]*{\scriptstyle 1})
 -[dr(.5)]\bullet([r(.1)]*{\scriptstyle 3})
 -[l]\bullet([l(.1)]*{\scriptstyle 2})
 ([ur(.5)]\bullet-?)
}}$
\dogap

\Section{Binary operations}

These are binary operations that can be performed over graphs.

\>BoxProduct( <G>, <H> ) O

Returns the box product, <G> $\square$ <H>, of two graphs <G> and <H> 
(also known as the cartesian product).

The box product is calculated as follows:

For each pair of vertices $g \in G, h \in H$ we create a vertex
$(g,h)$. Given two such vertices $(g,h)$ and $(g',h')$ they are
adjacent <iff> $g = g'$ and $h \sim h'$ or $g \sim g'$ and $h = h'$.

\beginexample
gap> g1:=PathGraph(3);g2:=CycleGraph(4);
Graph( Category := SimpleGraphs, Order := 3, Size := 2, Adjacencies := 
[ [ 2 ], [ 1, 3 ], [ 2 ] ] )
Graph( Category := SimpleGraphs, Order := 4, Size := 4, Adjacencies := 
[ [ 2, 4 ], [ 1, 3 ], [ 2, 4 ], [ 1, 3 ] ] )
gap> g1g2:=BoxProduct(g1,g2);           
Graph( Category := SimpleGraphs, Order := 12, Size := 20, Adjacencies := 
[ [ 2, 4, 5 ], [ 1, 3, 6 ], [ 2, 4, 7 ], [ 1, 3, 8 ], [ 1, 6, 8, 9 ], 
  [ 2, 5, 7, 10 ], [ 3, 6, 8, 11 ], [ 4, 5, 7, 12 ], [ 5, 10, 12 ], 
  [ 6, 9, 11 ], [ 7, 10, 12 ], [ 8, 9, 11 ] ] )
gap> VertexNames(g1g2);
[ [ 1, 1 ], [ 1, 2 ], [ 1, 3 ], [ 1, 4 ], [ 2, 1 ], [ 2, 2 ], [ 2, 3 ], 
  [ 2, 4 ], [ 3, 1 ], [ 3, 2 ], [ 3, 3 ], [ 3, 4 ] ]
\endexample



$$
\vcenter{\xymatrix{
   {\bullet} \\
   {\bullet} \ar@{-}[u]
}}
\square
\vcenter{\xymatrix{
   {\bullet} \\
   {\bullet} \ar@{-}[u]
}}
=
\vcenter{\xymatrix{
   {\bullet} \ar@{-}[r] \ar@{-}[d] & {\bullet} \ar@{-}[d]\\
   {\bullet} \ar@{-}[r] & {\bullet} 
}}
$$

\>TimesProduct( <G>, <H> ) O

Returns the times product of two graphs <G> and <H>, <G> $\times$ <H>
(also known as the tensor product).

The times product is computed as follows:

For each pair of vertices $g \in G, h \in H$ we create a vertex
$(g,h)$. Given two such vertices $(g,h)$ and $(g',h')$ they are
adjacent <iff> $g \sim g'$ and $h \sim h'$.

\beginexample
gap> g1:=PathGraph(3);g2:=CycleGraph(4);                              
Graph( Category := SimpleGraphs, Order := 3, Size := 2, Adjacencies := 
[ [ 2 ], [ 1, 3 ], [ 2 ] ] )
Graph( Category := SimpleGraphs, Order := 4, Size := 4, Adjacencies := 
[ [ 2, 4 ], [ 1, 3 ], [ 2, 4 ], [ 1, 3 ] ] )
gap> g1g2:=TimesProduct(g1,g2);         
Graph( Category := SimpleGraphs, Order := 12, Size := 16, Adjacencies := 
[ [ 6, 8 ], [ 5, 7 ], [ 6, 8 ], [ 5, 7 ], [ 2, 4, 10, 12 ], [ 1, 3, 9, 11 ], 
  [ 2, 4, 10, 12 ], [ 1, 3, 9, 11 ], [ 6, 8 ], [ 5, 7 ], [ 6, 8 ], [ 5, 7 ] ] )
gap> VertexNames(g1g2);                 
[ [ 1, 1 ], [ 1, 2 ], [ 1, 3 ], [ 1, 4 ], [ 2, 1 ], [ 2, 2 ], [ 2, 3 ], 
  [ 2, 4 ], [ 3, 1 ], [ 3, 2 ], [ 3, 3 ], [ 3, 4 ] ]
\endexample



$$
\vcenter{\xymatrix{
   {\bullet} \\
   {\bullet} \ar@{-}[u]
}}
\times
\vcenter{\xymatrix{
   {\bullet} \\
   {\bullet} \ar@{-}[u]
}}
=
\vcenter{\xymatrix{
   {\bullet} \ar@{-}[rd] & {\bullet} \ar@{-}[dl]\\
   {\bullet} & {\bullet} 
}}
$$

\>BoxTimesProduct( <G>, <H> ) O

Returns the boxtimes product of two graphs <G> and <H>, <G>
$\boxtimes$ <H> (also known as the strong product). 

The box times product is calculated as follows:

For each pair of vertices $g \in G, h \in H$ we create a vertex
$(g,h)$. Given two such vertices $(g,h)$ and $(g',h')$ such that
$(g,h) \neq (g',h')$ they are adjacent <iff> $g \simeq g'$ and $h
\simeq h'$. 

\beginexample
gap> g1:=PathGraph(3);g2:=CycleGraph(4);
Graph( Category := SimpleGraphs, Order := 3, Size := 2, Adjacencies := 
[ [ 2 ], [ 1, 3 ], [ 2 ] ] )
Graph( Category := SimpleGraphs, Order := 4, Size := 4, Adjacencies := 
[ [ 2, 4 ], [ 1, 3 ], [ 2, 4 ], [ 1, 3 ] ] )
gap> g1g2:=BoxTimesProduct(g1,g2);      
Graph( Category := SimpleGraphs, Order := 12, Size := 36, Adjacencies := 
[ [ 2, 4, 5, 6, 8 ], [ 1, 3, 5, 6, 7 ], [ 2, 4, 6, 7, 8 ], [ 1, 3, 5, 7, 8 ], 
  [ 1, 2, 4, 6, 8, 9, 10, 12 ], [ 1, 2, 3, 5, 7, 9, 10, 11 ], 
  [ 2, 3, 4, 6, 8, 10, 11, 12 ], [ 1, 3, 4, 5, 7, 9, 11, 12 ], 
  [ 5, 6, 8, 10, 12 ], [ 5, 6, 7, 9, 11 ], [ 6, 7, 8, 10, 12 ], 
  [ 5, 7, 8, 9, 11 ] ] )
gap> VertexNames(g1g2);                 
[ [ 1, 1 ], [ 1, 2 ], [ 1, 3 ], [ 1, 4 ], [ 2, 1 ], [ 2, 2 ], [ 2, 3 ], 
  [ 2, 4 ], [ 3, 1 ], [ 3, 2 ], [ 3, 3 ], [ 3, 4 ] ]
\endexample



$$
\vcenter{\xymatrix{
   {\bullet} \\
   {\bullet} \ar@{-}[u]
}}
\boxtimes
\vcenter{\xymatrix{
   {\bullet} \\
   {\bullet} \ar@{-}[u]
}}
=
\vcenter{\xymatrix{
   {\bullet} \ar@{-}[r] \ar@{-}[d] \ar@{-}[rd] & {\bullet}
   \ar@{-}[d] \ar@{-}[dl] \\ 
   {\bullet} \ar@{-}[r] & {\bullet} 
}}
$$

In the previous examples $k^2$ (<i.e.> the complete graph or order
two) was chosen because it better pictures how the operators work.

\>DisjointUnion( <G>, <H> ) O

Returns the disjoint union of two graphs <G> and <H>, <G> $\dot{\cup}$ <H>.

\beginexample
gap> g1:=PathGraph(3);g2:=PathGraph(2); 
Graph( Category := SimpleGraphs, Order := 3, Size := 2, Adjacencies := 
[ [ 2 ], [ 1, 3 ], [ 2 ] ] )
Graph( Category := SimpleGraphs, Order := 2, Size := 1, Adjacencies := 
[ [ 2 ], [ 1 ] ] )
gap> DisjointUnion(g1,g2);
Graph( Category := SimpleGraphs, Order := 5, Size := 3, Adjacencies := 
[ [ 2 ], [ 1, 3 ], [ 2 ], [ 5 ], [ 4 ] ] )
\endexample



$$
\vcenter{\xymatrix{
   {\bullet} \ar@{-}[d] & {\bullet} \ar@{-}[l] \\
   {\bullet} \ar@{-}[ur]
}}
\dot{\cup}
\vcenter{\xymatrix{
   {\bullet} \\
   {\bullet} \ar@{-}[u]
}}
=
\vcenter{
\hbox{
\xymatrix{
   {\bullet} \ar@{-}[d] & {\bullet} \ar@{-}[l] \\
   {\bullet} \ar@{-}[ur]
}
\enskip
\xymatrix{
   {\bullet} \\
   {\bullet} \ar@{-}[u]
}}}
$$

\>Join( <G>, <H> ) O

Returns the result of joining graph <G> and <H>, <G> + <H> 
(also known as the Zykov sum).

Joining graphs is computed as follows:

First, we obtain the disjoint union of graphs <G> and <H>. Second,
for each vertex $g \in G$ we add an edge to each vertex $h \in H$.

\beginexample
gap> g1:=DiscreteGraph(2);g2:=CycleGraph(4);
Graph( Category := SimpleGraphs, Order := 2, Size := 0, Adjacencies := 
[ [  ], [  ] ] )
Graph( Category := SimpleGraphs, Order := 4, Size := 4, Adjacencies := 
[ [ 2, 4 ], [ 1, 3 ], [ 2, 4 ], [ 1, 3 ] ] )
gap> Join(g1,g2);                           
Graph( Category := SimpleGraphs, Order := 6, Size := 12, Adjacencies := 
[ [ 3, 4, 5, 6 ], [ 3, 4, 5, 6 ], [ 1, 2, 4, 6 ], [ 1, 2, 3, 5 ], 
  [ 1, 2, 4, 6 ], [ 1, 2, 3, 5 ] ] )
\endexample



$$
\vcenter{\xymatrix{
   {\bullet} \ar@{-}[d] & {\bullet} \ar@{-}[l] \\
   {\bullet} \ar@{-}[ur]
}} \phantom{cm}
+ \phantom{cm}
\vcenter{\xymatrix{
   {\bullet} \\
   {\bullet} \ar@{-}[u]
}}
=
\vcenter{\xymatrix{
   {\bullet} \ar@{-}[d] & {\bullet} \ar@{-}[l] \ar@{-}[r] & 
              {\bullet} \ar@{-}[lld] \ar@(ul,ur)@{-}[ll] \\
   {\bullet} \ar@{-}[ur] \ar@{-}[rr] & & 
              {\bullet} \ar@{-}[u] \ar@{-}[llu] \ar@{-}[lu]
}}
$$

\>GraphSum( <G>, <L> ) O

Returns the lexicographic sum of a list of graphs <L> over a graph <G>.

The lexicographic sum is computed as follows:

Given <G>, with $Order(G)=n$ and a list of <n> graphs $L = [G_1, \ldots, G_n]$,
We take the disjoint union of $G_1,G_2, \ldots,G_n$ and then we add all the 
edges between $G_i$ and $G_j$ whenever $[i,j]$ is and edge of $G$.

If <L> contains holes, the trivial graph is used in place.

\beginexample
gap> t:=TrivialGraph;; g:=CycleGraph(4);;
gap> GraphSum(PathGraph(3),[t,g,t]);
Graph( Category := SimpleGraphs, Order := 6, Size := 12, Adjacencies := 
[ [ 2, 3, 4, 5 ], [ 1, 3, 5, 6 ], [ 1, 2, 4, 6 ], [ 1, 3, 5, 6 ], 
  [ 1, 2, 4, 6 ], [ 2, 3, 4, 5 ] ] )
gap> GraphSum(PathGraph(3),[,g,]);  
Graph( Category := SimpleGraphs, Order := 6, Size := 12, Adjacencies := 
[ [ 2, 3, 4, 5 ], [ 1, 3, 5, 6 ], [ 1, 2, 4, 6 ], [ 1, 3, 5, 6 ], 
  [ 1, 2, 4, 6 ], [ 2, 3, 4, 5 ] ] )
\endexample


\>Composition( <G>, <H> ) O

Returns the composition $G[H]$ of two graphs <G> and <H>.

A composition of graphs is obtained by calculating the GraphSum
of <G> with <Order(G)> copies of <H>, $G[H] = GraphSum(G, [H, \ldots, H])$.

\beginexample
gap> g1:=CycleGraph(4);;g2:=DiscreteGraph(2);;                  
gap> Composition(g1,g2);                      
Graph( Category := SimpleGraphs, Order := 8, Size := 16, Adjacencies := 
[ [ 3, 4, 7, 8 ], [ 3, 4, 7, 8 ], [ 1, 2, 5, 6 ], [ 1, 2, 5, 6 ], 
  [ 3, 4, 7, 8 ], [ 3, 4, 7, 8 ], [ 1, 2, 5, 6 ], [ 1, 2, 5, 6 ] ] )
\endexample








