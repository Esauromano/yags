% This file was created automatically from kernel.msk.
% DO NOT EDIT!
\Chapter{Kernel}

The kernel contains definitions and the core functions that handle
graphs.

\Section{Categories}

Using {\GAP} category facilities \YAGS \ defines a hierarchy of
graphs. The categories defined are as follows.

\>Graphs( ) C

Graphs are the base category used by \UVA. This category contains
all graphs that can be represented in \UVA. 



Among them we can find:

$$
\xymatrix{
   {\bullet} \ar@(dr,dl)[d] \ar[dr] \\
   {\bullet} \ar@(ul,ur)[u] \ar[r] & {\bullet}
} 
\ 
\xymatrix{
   {\bullet} \ar@{-}[dr] & {\bullet} \ar[dl]\\
   {\bullet} & {\bullet} \ar@(ur,dr)[]\ar[l]
} 
$$
$$
\xymatrix{
   {\bullet} \ar@(ul,dl)[] \ar[dr] & & {\bullet} \ar@(l,d)[dl]\\
   & {\bullet} \ar@(r,u)[ur] & 
} 
\xymatrix{
   & {\bullet} \ar@{-}[dl] \ar@{-}[dr] & & {\bullet} \ar@{-}[dl] \\
   {\bullet} & & {\bullet} \ar@{-}[dr] \\
   & {\bullet} \ar@{-}[ur] \ar@{-}[ul] & & {\bullet} \ar@{-}[uu] \\
}
$$

\>LooplessGraphs( ) C

Loopless Graphs are graphs which have no loops.



A loop is an arrow that starts and finishes on the same
vertex. 

$$
\xymatrix{
   {\bullet} \ar@(ur,dr)[]
} 
$$

Loopless graphs have no such arrows.

$$
\xymatrix{
   {\bullet} \ar[dr] & & {\bullet} \ar@(l,d)[dl]\\
   & {\bullet} \ar@(r,u)[ur] & 
} 
$$

\>UndirectedGraphs( ) C

Undirected Graphs are graphs which have no directed arrows.



Given two vertex $i,j$ in graph $G$ we will say that graph $G$ has an
*edge* $\{i,j\}$ if there is an arrow $(i,j)$ and and arrow $(j,i).$ 

$$
\xymatrix{
   {\bullet} \ar@(ur,dr)[r] & {\bullet} \ar@(dl,ul)[l]
} 
\xymatrix{
   {\bullet} \ar@{-}[r] &  {\bullet}
} 
$$

Undirected graphs have no arrows but only edges.

$$
\xymatrix{
   {\bullet} \ar@(ul,dl)[] \ar@(r,d)[dr] & & {\bullet} \ar@(l,d)[dl]\\
   & {\bullet} \ar@(r,u)[ur] \ar@(l,u)[ul] & 
} 
\equiv\hskip20pt
\xymatrix{
   {\bullet} \ar@(ul,dl){-} \ar@{-}[dr] & & {\bullet} \ar@{-}[dl]\\
   & {\bullet} & 
}  
$$

\>OrientedGraphs( ) C

Oriented Graphs are graphs which have arrows in only one direction
between any two vertices. 



Oriented graphs have no edges but only arrows. 

$$
\xymatrix{
   {\bullet} \ar[dr] & & {\bullet} \ar[ll]\\
   & {\bullet} \ar[ur] & 
} 
$$

\>SimpleGraphs( ) C

Simple Graphs are graphs with no loops and undirected.


$$
\xymatrix{
   & {\bullet} \ar@{-}[dl] \ar@{-}[dr] & & {\bullet} \ar@{-}[dl] \\
   {\bullet} & & {\bullet} \ar@{-}[dr] \\
   & {\bullet} \ar@{-}[ur] \ar@{-}[ul] & & {\bullet} \ar@{-}[uu] \\
}
$$

Figure shows the relationship among categories.

$$
\pstree[nodesep=5pt]{\Tr{Graphs}}
{
  \pstree{\Tr{Loopless}}
  {
         \Tr{Oriented}
         \Tr[name=S]{Simple Graphs}
  }
  \Tr[name=U]{Undirected}
}
\ncline[nodesep=5pt]{S}{U}
$$

This relationship is important because when a graph is created it is
normalized in the category it belongs. For instance, if we create a
graph such as 
$$
\xymatrix{
   {\bullet} \ar[dr] & & {\bullet} \ar[ll]\\
   & {\bullet} \ar[ur] & 
}
$$
as a simple graph \YAGS\  will normalize the graph as
$$
\xymatrix{
   {\bullet} \ar@{-}[dr] & & {\bullet} \ar@{-}[ll]\\
   & {\bullet} \ar@{-}[ur] & 
}
$$
For further examples see the following section.


\Section{Default Category}

There are several means to specify the category in which a new graph
will be created. There exists a <DefaultCategory> which tells \YAGS\  to
which category belongs any new graph by default. The <DefaultCategory> can be
changed using the following function.

\>SetDefaultGraphCategory( <C> ) F

Sets category C to be the default category for graphs. The default
category is used, for instance, when constructing new graphs. 

\beginexample
SetDefaultGraphCategory(Graphs);
G:=RandomGraph(4);
Graph( Category := Graphs, Order := 4, Size := 8, Adjacencies :=
[ [ 3, 4 ], [ 4 ], [ 1, 2, 3, 4 ], [ 2 ] ] )
\endexample

RandomGraph creates a random graphs belonging to the category
graphs. The above graph has loops which are not permitted in
simple graphs.

\beginexample
SetDefaultGraphCategory(SimpleGraphs);
G:=CopyGraph(G);
Graph( Category := SimpleGraphs, Order := 4, Size := 5, Adjacencies :=
[ [ 3, 4 ], [ 3, 4 ], [ 1, 2, 4 ], [ 1, 2, 3 ] ] )
\endexample

Now G is a simple graph.



In order to handle graphs with different categories there two
functions available.

\>GraphCategory( [<G>, ... ] ) F

Returns the minimal common category to a list of graphs. See
Section "Categories" for the relationship among categories.
 
If the list is empty the default category is returned. 



\>TargetGraphCategory( [<G>, ... ] ) F

Returns the category which will be used to process a list of
graphs. If an option category has been given it will return that
category. Otherwise it will behave as Function <GraphCategory>
("GraphCategory"). See Section "Categories" for the relationship
among categories. 



Finally we can test if a single graph belongs to a given category.

\>in( <G>, <C> ) O

Returns `true' if graph <G> belongs to category <C> and `false' otherwise.



\Section{Atributes and properties of graphs}

The following are functions to obtain atributes and properties of
graphs.

\>AdjMatrix( <G> ) A

The adjacency matrix of graph <G>.


\>Order( <G> ) A

The order, <i.e.> number of vertices, of graph <G>.


\>Size( <G> ) A

The size, <i.e.> number of arrows or edges of graph <G>.


\>VertexNames( <G> ) A

A list of all vertices in graph <G>.


\>IsCompleteGraph( <G> ) A
\>QtfyIsCompleteGraph( <G> ) P

The attribute form is `true' if graph <G> is complete.
The property form measures how far graph <G> is from being complete.


\>IsLoopless( <G> ) A
\>QtfyIsLoopless( <G> ) P

The attribute form is `true' if graph <G> has no loops.
The property form measures how far graph <G> is from being
loopless, <i.e.> the number of loops in <G>.


\>IsUndirected( <G> ) A
\>QtfyIsUndirected( <G> ) P

The attribute form is `true' if graph <G> has only edges and no arrows.
The property form measures how far graph <G> is from being
undirected, <i.e.> the number of arrows in <G>.


\>IsOriented( <G> ) A
\>QtfyIsOriented( <G> ) P

The attribute form is `true' if graph <G> has only arrows.
The property form measures how far graph <G> is from being
oriented, <i.e.> the number of edges in <G>.



\Section{Information about graphs}

The following functions give information regarding graphs.

\>IsSimple( <G> ) O

Returns `true' if the graph <G> is simple regardless of its category.


\>QtfyIsSimple( <G> ) O

Returns how far is graph <G> from being simple.


\>Adjacency( <G>, <V> ) O

Returns the adjacency list of vertex <V> in <G>.


\>Adjacencies( <G> ) O

Returns the adjacencies list of graph <G>


\>VertexDegree( <G>, <V> ) O

Returns the degree (number of adjacencies) of vertex <V> in Graph <G>. 


\>VertexDegrees( <G> ) O

Returns a list with the degrees of all vertices in graph <G>.


\>Edges( <G> ) O

Returns a list of all edges in graph <G>. Each edge is represented
as $[i,j]$ indicating there is an edge from vertex <i> to vertex <j>. 



\Section{Creating graphs}

The following functions create new graphs from a variety of sources.

\>Graph( <R> ) O

Creates a graph from the record <R>. There are two representations
from which a graph can be created:
\beginlist
\item{1.}
From an adjacencies list.
\item{2.}
From an adjacency matrix.
\endlist
The record must, therefore, provide a field <Adjacencies> containing
the list of adjacencies for each vertex or alternatively a field
<AdjMatrix> with the adjacency matrix. Additionaly we must provide
the category, or categories, to which the new graph belongs. 


\>GraphByAdjMatrix( <M> ) F

Creates a graph from an adjacency matrix <M>. The matrix <M> must
be a square boolean matrix. The category to which the graph
created belongs is the default category. For more information on
categories see Section Categories("Categories").


\>GraphByAdjacencies( <A> ) F

Creates a graph from a list of adjacencies <A>. The category to
which the graph created belongs is the default category. For more
information on Section Categories("Categories").


\>GraphByCompleteCover( <C> ) F

Creates a complete cover graph os size <C>.


\>GraphByRelation( <V>, <R> ) F
\>GraphByRelation( <N>, <R> ) F

Creates a graph from a relation.


\>IntersectionGraph( <L> ) F

Creates a graph from list <L> where <L> is an intersection list.



The following functions create graphs from existing graphs

\>CopyGraph( <G> ) O

Creates a fresh copy of graph <G>.


\>InducedSubgraph( <G>, <V> ) O

Creates an induced graph from graph <G> using only vertices <V>.


\>RemoveVertices( <G>, <V> ) O

Creates a graph from graph <G> removing vertices <V>.


\>AddEdges( <G>, <E> ) O

Creates a graph from graph <G> adding the set of edges <E>.


\>RemoveEdges( <G>, <E> ) O

Creates a graph from graph <G> removing edges <E>.







